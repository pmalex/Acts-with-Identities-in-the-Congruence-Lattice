\documentclass{birkau}

\usepackage{amsmath,amssymb,url}

%%%%%%%%%%%%%%%%%%%%%%%%%%%%%%%%%%%%%%%%

\numberwithin{equation}{section}

\theoremstyle{plain}
\newtheorem{theorem}{Theorem}[section]
\newtheorem{lemma}[theorem]{Lemma}
\newtheorem{proposition}[theorem]{Proposition}
\newtheorem{corollary}[theorem]{Corollary}

\theoremstyle{definition}
\newtheorem{definition}[theorem]{Definition}
\newtheorem{notation}[theorem]{Notation}
\newtheorem{remark}[theorem]{Remark}
\newtheorem*{note}{Note} %% Use this construction for an unnumbered declaration.

%%%%%%%%%%%%%%%%%%%%%%%%%%%%%%%%%%%%%%%%%%%%%%%
%% Macroses
%%%%%%%%%%%%%%%%%%%%%%%%%%%%%%%%%%%%%%%%%%%%%%%

\DeclareMathOperator{\Con}{Con}
\DeclareMathOperator{\Eq}{Eq}
\DeclareMathOperator{\Var}{Var}

%%%%%%%%%%%%%%%%%%%%%%%%%%%%%%%%%%%%%%%%%%%%%%%%%%%%%%%

\begin{document}
	
    \title[Acts with Identities in the Congruence Lattice]{Acts with Identities in the Congruence Lattice}

    %% First author: in the order \author, \address, \urladdr, \email
    \author[I.B. Kozhuhov]{I.B. Kozhuhov}
    \address{National Research University MIET, \\
    Faculty of Mechanics and Mathematics of Lomonosov Moscow State University, \\
    Center of Fundamental and Applied Mathematics of Lomonosov MSU \\
    Moscow\\Russia}
    %\urladdr{http://www.cs.uwinnebago.edu/homepages/~menuhin}
    \email{kozhuhov\_i\_b@mail.ru}

    %% Second author: in the order \author, \address, \urladdr, \email
    \corrauthor[A.M. Pryanichnikov]{A.M. Pryanichnikov}
    \address{Lomonosov Moscow State University\\Moscow\\Russia}
    \email{genary@ya.ru}
    %% AMS subject classification; see http://www.ams.org/msc
    %% List classification codes in order of relevance
    \subjclass{06B20, 08A30, 20M35}

    \thanks{Research supported by a grant of the Center Fund. Appl. Math. of Lomonosov MSU}
	
    %% Key words and phrases
    \keywords{act over semigroup, congruence lattice, lattice identity}

    \begin{abstract}
    We prove that for any act $X$ over a finite semigroup $S$, the congruence lattice $\Con X$ embeds the lattice $\Eq M$ of all equivalences of an infinite set $M$ if and only if $X$ is infinite. Equivalently: for an act $X$ over a finite semigroup $S$, the lattice $\Con X$ satisfies a non-trivial identity if and only if $X$ is finite. Similar statements are proved for an act with zero over a completely 0-simple semigroup $\mathcal M^0(G,I,\Lambda,P)$ where $|G|,|I| <\infty$. We construct examples that show that the assumption $|G|,|I| <\infty$ is essential.
    \end{abstract}
	
	\maketitle
	
	\section{Introduction}
	
	The congruence lattice $\Con A$ of a universal algebra $ A $ is an important characteristic of $A$. The smallest element of this lattice is the equality relation $ \Delta_A = \{ (a,a) \mid a \in A \} $, and the greatest one is the universal relation $ \nabla_A = A \times A $. The lattice $\Con A$ is a complete sublattice of the lattice $\Eq A$ of all equivalence relations on the set $A$.
    One of the research areas of general algebra is investigating algebras with certain conditions on congruences.	For example, the condition of triviality ($ \Con A = \{ \Delta_A, \nabla_A \} $) defines simple algebras (simple groups, rings, congruence-simple semigroups, etc.), maximality or minimality conditions define noetherian and artinian algebras.
	
	Paper \cite{resh} describes the class of algebras, opposite to the class of simple algebras, namely, algebras in which all equivalence relations are congruences (i.e. $ \text{Con}A = \text{Eq}A $).
    There are many works on subdirectly irreducible algebras, i.e. algebras $A$ such that either $ |A| = 1 $ or the lattice Con$A$ contains a smallest element different from $ \Delta_A $.
	
	The universal algebras with modular, or distributive or chained congruence lattice also attracted the attention of specialists. In particular, there are papers devoted to distributive and chained rings and modules and also acts (over semigroups) with distributive or modular congruence lattice~\cite{step,hal3}. It should be noted that although an act over a semigroup is an analog of the module over a ring, the congruence lattice of a module (i.e. the lattice of submodules) is always modular, however, for the congruence lattice of an act, modularity is a rare phenomenon. Distributive and modular lattices form varieties, which are determined by the identities $ (x \vee y) \wedge z = (x \wedge z ) \vee (y \wedge z) $ and $ x \wedge ( y \vee z ) = x \wedge (( y \wedge (x \vee z)) \vee z ) $ respectively (see~\cite[Chapter 4, Theorem 1.1]{gretz}).  Chains form a class of lattices, which is not a variety, but is closed with respect to sublattices and homomorphic images.
	
    In \cite{memoirs}, the congruence lattices and their identities are investigated. \cite{jipsen} studies varieties of lattices (and hence the lattice identities).

    It is natural to consider the classes of algebras whose congruence lattices satisfy a lattice identity.
    Note that the existence of a non-trivial identity of $\Con A$ can be considered as a finiteness condition on $A$.

    One can found the basic notions and facts of the theory of acts in \cite{kilp}, semigroup theory -- in \cite{cliff}, lattice theory -- in \cite{gretz}, universal algebra -- in \cite{burris} and  \cite{kon}, lattice varieties -- in \cite{jipsen}. 	
    In what follows, $\mathcal{V}$ will denote the \textit{variety of all lattices}, and $\Var L$ \textit{the variety generated by the lattice $L$}. Moreover, $\mathcal{M}(G,I,\Lambda,P)$ and $\mathcal{M}^0(G,I,\Lambda,P)$ will denote a \textit{Rees matrix semigroup} and a \textit{regular Rees matrix semigroup with zero} (see~\cite[Chapter 2]{cliff}). A lattice identity is called \textit{non-trivial} if it holds in not-all lattices.
	
	The following facts on the lattice varieties are known. They may be obtained by means of statements \cite[Corollary 3.14]{kon}, \cite[Theorem 1.28]{free_lattices}, \cite{sachs}. In sequence, they will be used without references.

{\bf Fact 1.} In a finite lattice, a non-trivial identity holds.

{\bf Fact 2.} In the lattice $\Eq M$ where $M$ is an infinite set, only trivial identities hold.
	
{\bf Fact 3.} If a lattice $L$ contains $\Eq M$ as a sublattice where $M$ is an infinite set, then $\Var L = \mathcal{V}$.

	
	The main results of this work are the following statements.
	
	\begin{theorem} \label{t01}
	    Let $X$ be an act over a finite semigroup. Then the congruence lattice $\Con X$ embeds the lattice of all equivalences on an infinite set iff $X$ is infinite.
	\end{theorem}
	
	The theorem implies immediately
	
	\begin{corollary}
	    For an act $X$ over a finite semigroup, the lattice $\Con X$ satisfies a non-trivial lattice identity iff $X$ is finite.
	\end{corollary}
	
	\begin{theorem} \label{t02}
	    Let $S = \mathcal{M}^0(G,I,\Lambda,P)$ be a completely 0-simple semigroup and $|G| < \infty,\, |I| < \infty $. Then, for any act $X$ with zero over the semigroup $S$, the following holds: the congruence lattice $\Con X$ embeds the lattice of all equivalences on an infinite set iff $X$ is infinite.
	\end{theorem}
	
	Theorem \ref{t02} yields the corresponding result for acts over completely simple Rees matrix semigroups as well.
	
	In the same manner, the theorem implies
	
	\begin{corollary}
	    If $|G|,|I| < \infty$ then $\Con X$ satisfy a non-trivial identity iff $X$ is finite.
	\end{corollary}
	
	If $I$ is an infinite set, then the statement of the Theorem~\ref{t02} is invalid. We construct two examples of infinite acts $X$ which are \textit{congruence-simple} (i.e. $\Con X = \{ \Delta, \nabla \}$). The first one is an act $X$ with zero over a semigroup $\mathcal{M}^0(\{e\},\mathbb{N},\mathbb{N},P)$, the second one is an act over a semigroup $\mathcal{M}(S_3,\mathbb{N}_0,\mathbb{N}_0,P)$ where $S_3$ is the group of permutations of a 3-element set. In both examples, $X$ is an infinite act but $\Con X$ is finite and therefore $\Con X$ satisfies a non-trivial identity.
	
	\begin{proposition} \label{pr01a}
    	Every minimal right ideal $X$ of $S = \mathcal{M}^0(\{1\},\mathbb{N},\mathbb{N},I) $ for $\mathbb{N} \times \mathbb{N}$-identity matrix $I$ forms a congruence-simple infinite act.
    	
	   % There exists a completely 0-simple semigroup $ S = $ \newline $ = \mathcal{M}^0(G,I,\Lambda,P) $ with $|G| = 1$ and an infinite act $X$ with zero over $S$ such that the lattice $\Con X$ is two-element.
	\end{proposition}
	
	\begin{proposition} \label{pr2.1}
	    There exists a congruence-simple infinite act $X$ over \newline $\mathcal{M}(S_3,\mathbb{N},\mathbb{N},P) $ for some matrix $P$.
	\end{proposition}

\section{Acts over semigroups}	

	\textit{An act over a semigroup} is the set $X$ on which the semigroup $S$ acts, i.e. a mapping $ X \times S \rightarrow X,\ (x,s) \mapsto xs $ is defined, satisfying the condition $ x(st)=(xs)t $ for all $x\in X,\ s,t\in S$ (see~\cite{kilp}). An act can be considered as a unary algebra, i.e. an algebra in which all operations are unary (operations of an act $X$ over a semigroup $S$ are multiplications by elements of a semigroup, i.e. $ x \mapsto xs \ (s\in S) $).
	
	If an act $X$ is a union of its subacts $ X_i\ (i \in I) $ and $ X_i \cap X_j = \emptyset $ for $i \neq j$, then we call $X$ a \textit{coproduct} of acts $X_i$ and write $ X = \coprod_{i\in I} X_i $. For any semigroup $S$, we denote by $S^1$ the smallest semigroup with unity containing $S$, i.e.
	$$ S^1 =
		\begin{cases}
			S & \text{if $S$ has a unity,}\\
			S \cup \{1\} & \text{otherwise.}
		\end{cases}
	$$
	Let $X$ be an act over a semigroup $S$. For $ x,y \in X$, put $ x \leqslant y \Leftrightarrow x \in yS^1 $. Obviously, the relation $\leqslant$ is a quasi-order on the set $X$. An act $X$ is called \textit{connected} if for any $x,y\in X$ there exists a sequence of elements $x_0,x_1,\ldots,x_{2k}\in X$ such that $$ x \geqslant x_0 \leqslant x_1 \geqslant x_2 \leqslant \ldots \geqslant x_{2k} \leqslant y. $$ It is easy to see that the connectivity of the act $X$ over the semigroup $S$ is exactly the connectivity of the graph with vertex set $X$ and edges $ (x,xs) $, where $ x \in X,\ s \in S $ and $x \neq xs$. Moreover, every act is a coproduct of connected subacts (the connected components).
	
	A \textit{zero} of an act $X$ over semigroup $S$ is an element $z \in X$ such that $zs=z$ for all $s \in S$.
	Let $X$ be an act over a semigroup $S$ and $Y$ be a subact. The congruence $\rho_Y = (Y \times Y) \cup \Delta_X$ is called the \textit{Rees congruence} induced by $Y$. For a factor act ${X}/{\rho_Y}$, we will also use the term $X/Y$. The following statement is well known, we give the proof only for completeness.
	
	\begin{lemma} \label{lemma:01}
	    Let $X$ be an act and $Y$ be its subact. Then the lattices $\Con Y$ and $\Con {X}/{Y}$ are isomorphically embedded into the lattice $\Con X$.
	\end{lemma}
	\begin{proof}
	    It is easy to verify that the map $\rho \mapsto \rho \cup \Delta_X$ is an isomorphic embedding of $\Con Y$ into $\Con X$.
		
	The statement that $\Con {X}/{Y}$ is a sublattice of $\Con X$ follows from a general algebraic fact: if $\rho$ is a congruence of the algebra $A$, then there exists a one-to-one correspondence between the congruences of the algebra ${A}/{\rho}$ and those congruences on $A$ that contain $\rho$; this correspondence preserves the operations $\vee$ and $\wedge$, so the lattice $\Con {A}/{\rho}$ is isomorphic to the interval $[\rho, \nabla_A]$ of the lattice $\Con A$ \cite[Theorem 6.20]{burris}.
    \end{proof}
	
	\section{Acts over finite semigroups}
	
	Let $X$ be an act over a semigroup $S$. The equivalence relation $\sim$ on $X$ and the order relation on $X/{\sim}$ are determined by the quasi-order $\leqslant$ in the standard way. Namely, $$ x \sim y \Leftrightarrow x \leqslant y \wedge y \leqslant x. $$ and $\overline{x} \leqslant \overline{y} \Leftrightarrow x \leqslant y $ (if $\overline{x}, \overline{y}$ are ${\sim}$-classes containing $x,y$ resp.).
	For $x,y \in X$ we put $$ x < y \Leftrightarrow x \leqslant y \wedge x \nsim y. $$
	
	\begin{lemma} \label{lemma:03}
	    The relation $<$ on $X$ is transitive.
	\end{lemma}
	\begin{proof}
	    Let $x < y$ and $y < z$. Then $x \leqslant y$ and $y \leqslant z$. Because of the transitivity of the relation $\leqslant$, we have $x \leqslant z$. Assume that $x \sim z$, then $z \leqslant x$. Therefore $x \leqslant y \leqslant z \leqslant x$, i.e. $x \sim y$, which contradicts the assumption.
	\end{proof}
	
	\begin{lemma} \label{lemma:04}
	    Let $X$ be an act over a finite semigroup $S$ with $|S| = n$. If $x_0 < x_1 < \ldots < x_k$ is a sequence of elements of $X$ then $k \leqslant n$.
	\end{lemma}
	\begin{proof}
	    Let $k > n$. We have: $$x_0 = x_1 s_0,\ x_1 = x_2s_1, \ldots , x_{k-1} = x_k s_{k-1}$$ for some $s_0,\ldots,s_{k-1} \in S^1$. As $x_i \neq x_j$ for $i \neq j$, then $ s_0,\ldots,s_{k-1} \in S$. Obviously,
		\begin{gather}
			x_i = x_k s_{k-1} s_{k-2} \ldots s_i \label{lf0}
		\end{gather}
		for all $i = 0,1,\ldots,k-1$. The elements $s_{k-1},s_{k-1}s_{k-2},\ldots,s_{k-1}s_{k-2}\ldots s_1s_0$ belong to the semigroup $S$. As $|S| = n$ and $k>n$, they cannot have different elements. Therefore $s_{k-1}s_{k-2}\ldots s_i = s_{k-1}s_{k-2}\ldots s_j$ for some $i\neq j$. We may assume that $i < j$. The equation~\eqref{lf0} shows that $x_i = x_j$ which contradicts $x_j < x_i$.
	\end{proof}
	
	This lemma allows us to introduce the concept of length. Let $X$ be an act over a finite semigroup $S$ of size $n$. Put $$ Z_0 = \{ x \in X \mid \forall y \in X \ y \leqslant x \rightarrow y \sim x \}.$$ The \textit{length} $l(x)$ \textit{of an element} $x \in X$ is the largest number $k$ such that there exists a chain of elements $$x_0 < x_1 < \ldots < x_k=x.$$ As this chain is the longest, then $x_0 \in Z_0$. By Lemma~\ref{lemma:04} $k \leqslant n$. The length $l(x)$ of an act $X$ is a number $l(X) = \max \{ l(x) \mid x \in X\}$. Thus, $l(x) \leqslant n$ for all $x \in X$. Obviously $l(x) = 0 \Leftrightarrow x \in Z_0$. It is clear that $l(X) \leqslant n$.
	
	\begin{proof}[Proof of Theorem~\ref{t01}]
	    \textit{Sufficiency.} Since $X$ is finite, the lattice $\Con X$ is also finite. Therefore $\Con X$ satisfies a non-trivial identity.
	
		\textit{Necessity.} We should prove that if $X$ is infinite, the lattice $\Con X$ does not satisfy a non-trivial identity. The proof is implemented by induction on the length $l(X)$ of the act $X$.
		
		\textit{Basis of induction.} Let $l(X) = 0$. Then $X = Z_0$ and we have: $$ \forall x \in X \ \forall s \in S \ \exists t \in S \ \ xst = x. $$ In this case, the relation $\sim$ is a congruence and its classes are the sets $xS^1\ (x \in X)$. These sets are finite, so ${X}/{\sim}$ is an infinite act consisting only of zeros. It follows that $\Con({X}/{\sim})=\Eq({X}/{\sim})$, and $\Con({X}/{\sim})$ does not satisfy any non-trivial lattice identity. It follows from Lemma~\ref{lemma:01} that the lattice $\Con X$ also does not satisfy any non-trivial identity.
		
		\textit{Step of induction.} Let $l(X) = m > 0$. If $Z_0$ is an infinite subact, then the lattice $\Con Z_0$ is not contained in any proper subvariety of the variety $\mathcal{V}$. By Lemma~\ref{lemma:01}, the lattice $\Con X$ contains the lattice $\Con Z_0$, therefore the lattice $\Con X$ does not satisfy a non-trivial identity either.
		
		In the following we assume that $|Z_0| < \infty$. Put $X'={X}/{Z_0}$. Then $X'$ is an infinite act with the unique zero $z_0$. Clearly,  $l(z_0) = 0$ and $l(x) > 0$ for $x \neq z_0$.
		
		Let $Z_1 = \{x \in X' \mid \forall y  \in X' \ y < x \rightarrow y = z_0\}$. Clearly $z_0 \in Z_1$. Check that $Z_1$ is a subact. Let $x \in Z_1$, $s \in S$. If $x = z_0$ then $xs = z_0$. Let $x \neq z_0$. We have: $xs \leqslant x$. If $xs = z_0$ then $xs \in Z_1$. If $xs \neq z_0$ then $xs \not < x$. Hence $xs \sim x$. Let $y < xs$. Then $y \leqslant x$. If $y < x$ then $y = z_0$. Thus, $xs \in Z_1$.
		
		Suppose $ Z_1 $ is infinite. Consider the following relations on the act $Z_1$: $$ \alpha = \{ (x,y) \in Z_1 \times Z_1 \mid xS^1 = yS^1 \}, $$ $$ \beta = \{ (x,y) \in Z_1 \times Z_1 \mid \forall s \in S^1 \ xs = z_0 \leftrightarrow ys = z_0 \}. $$
		
		Clearly, the relation $\alpha$ is the same as $\sim$ on $Z_1$. Let us prove that $\beta$ is of finite index. Indeed, $\beta = \cap \{ \beta_s \mid s \in S^1 \}$ where $\beta_S = \{ (x, y) \in Z_1 \times Z_1 \mid xs = z_0 \leftrightarrow ys = z_0 \}$. As the equivalences $\beta_s$ are of index 2, then the index of $\beta$ is less or equal to $2^{n + 1}$ (since $|S| = n$). Now we show that the relation $\alpha \cap \beta$ is a congruence relation. Let $s \in S$ and $(x,y) \in \alpha \cap \beta$. If $xs = z_0$, then because of the inclusion $(x,y) \in \beta$, we also have $ys = z_0$. Hence $(xs,ys) \in \alpha \cap \beta$. Let $xs \neq z_0$. Then also $ ys \neq z_0 $. We have: $xs \leqslant x$. Since $xs \neq z_0$, then $xs \not < x$, hence $xs \sim x$. Similarly, $ys \sim y$. As $x \sim y$, then $ xs \sim ys$, i.e. $(xs,ys) \in \alpha$. We shall prove that $(xs,ys) \in \beta$. Since $xs \sim ys$ then $xs \cdot 1 = z_0 \leftrightarrow ys \cdot 1 = z_0$. Check that also $xs \cdot t = z_0 \leftrightarrow ys \cdot t = z_0$ for $t \in S$. Let $xst = z_0$. Since $(x,y) \in \beta$, then $yst = z_0$. Now it is clear that $(xs,ys) \in \beta$. Thus, $\alpha \cap \beta$ is a congruence on $Z_1$.
		
        As $Z_1$ is infinite and $\beta$ is of finite index then there is an infinite $\beta$-class $K$. As $(\alpha \cap \beta)$-classes are finite then $K$ contains infinitely many $(\alpha \cap \beta)$-classes. Put $\overline Z_1 = Z_1/(\alpha \cap \beta)$, $\overline K=K/(\alpha \cap \beta)$ (it is not necessarily an act). The set $\overline K$ is infinite. Define a mapping $\overline K \xrightarrow{\varphi}\Eq {S^1} \times 2^{S^1}$ as follows. For $\overline y \in \overline K$ we put $\varphi ( \overline y)=(\sigma(\overline y), A(\overline y))$ where $$ \sigma (\overline y) = \{ (s,t) \in S^1 \times S^1 \mid \overline y s=\overline y t\}, \quad A(\overline y) = \{ s\in S^1 \mid \overline y s =z_0 \}.$$ Here $ \sigma (\overline y)$ is a right congruence on the semigroup $S^1$ and $A(\overline y)$ is a right ideal of $S^1$. As $\overline K$ is infinite and $\Eq{S^1} \times 2^{S^1}$ is finite then there is an infinite class $P$ of the relation ${\rm ker}\varphi$. Thus, $\sigma (\overline y) =\sigma (\overline y')$ for all $\overline y, \overline y' \in P.$

        Consider a set $U=PS^1\cup\{z_0\}.$ Clearly, $U$ is a subact of $\overline Z_1.$ Let's prove that $\Eq P$ is embedded into $\Con U$.

        Let $\tau \in \Eq P.$ Put $\rho(\tau) = \{(\overline y_1s,\overline y_2s) \mid s\in S^1, (\overline y_1, \overline y_2)\in \tau\}\cup \{(z_0,z_0)\}.$ Check that $\rho(\tau)\in \Con U.$ The reflexivity and the symmetry of $\rho(\tau)$ are obvious. Now let's show that the transitivity also holds. Let $(u,v)$, $(v,w)\in \rho(\tau)$.
        Note that $(z_0,u) \in \rho(\tau)$ implies $u=z_0$. Indeed, if $u\ne z_0$ then $z_0=\overline y_1s,$ $u=\overline y_2s$ for some $\overline y_1, \overline y_2 \in P,$ $s\in S^1.$ As $\overline y_1, \overline y_2 \in P$ then $A(\overline y_1)=A(\overline y_2),$ and hence $\overline y_2s=z_0$, which is impossible.
        Thus, we may think that $u,v,w \ne z_0.$ Then $u=\overline y_1s$, $v=\overline y_2s=\overline y_3t$, $w=\overline y_4t$ for some $\overline y_1, \overline y_2, \overline y_3, \overline y_4 \in P$, $s,t\in S^1$ where $(\overline y_1, \overline y_2)$, $(\overline y_3, \overline y_4)\in \tau.$ This way we have $\overline y_2s=\overline y_3t \ne z_0.$ The element $\overline y_2 \in \overline K$ is a class containing some element $y_2\in K$. Similarly, $\overline y_3$ is a class of $y_3 \in K.$ As $y_2s\ne z_0$ then $(y_2,y_2s)\in\alpha$. Likewise $(y_3,y_3t)\in \alpha.$ As $y_2,y_3 \in K$ then $(y_2,y_3)\in \beta.$ Therefore $\overline y_2 =\overline y_3$. We have $\overline y_2s=\overline y_2t.$ It implies that $(s,t)\in \sigma (\overline y_2).$ As $\sigma(\overline y_2)=\sigma(\overline y_4)$ then $\overline y_4s=\overline y_4t.$ As $(\overline y_1, \overline y_2)$, $(\overline y_3, \overline y_4) \in \tau$ and $\overline y_2=\overline y_3$ then $(\overline y_1, \overline y_4) \in \tau$. Moreover, $(\overline y_1s, \overline y_4t)=(\overline y_1s, \overline y_4s)$. Therefore $(u,w)\in \rho(\tau)$. Thus, $\rho(\tau)$ is transitive. The implication $(p,q)\in \rho(\tau)\to (ps,qs) \in\rho(\tau)$ (for $(p,q)\in\tau$, $s\in S^1$) is obvious. Therefore, $\rho(\tau)$ is the congruence on $U$.

        Consider the mapping $\Eq P \to \Con U$, $\tau \mapsto \rho(\tau)$. It can be easily checked that this mapping is a lattice embedding $\Eq P$ into $\Con U$. Let's denote the isomorphic embedding of a lattices by the symbol $\underset\to \subset$. Then by Lemma \ref{lemma:01} we have $\Con U \underset \to \subset \Con \overline Z_1 \underset \to \subset \Con Z_1 \underset\to \subset \Con X$. Therefore $\Con X$ embeds $\Eq P$.
		
		To complete the proof of the theorem, we need to consider the case when $Z_1$ is a finite set. Put $X'' = X'/{Z_1}$.
		
		Now we prove that $l(X'') < m$. Let $ x_0 < x_1 < \ldots < x_k $ be the longest chain in $X''$. Assume that $k \geqslant m$ and lets show that this is a contradiction. Since $X'' = (X' \setminus Z_1) \cup \{0\}$ then $ x_1 \neq Z_1$. By definition of $Z_1$ there exists $y \in X'$ such that $y < x_1$ and $y \neq z_0$. If $y \notin Z_1$ then $z < y$ for some $z \neq z_0$, and we obtain the chain $z < y < x_1 < \ldots < x_k $ elements from $X$ which shows that $l(X) \geqslant k+1 > m$. This is a contradiction with the condition. So $y \in Z_1$ and $y \neq z_0$ in $X'$, i.e. $y \not \in Z_0$ in $X$. As $y \notin Z_0$, then $ z < y$ for some $z \in X$. We again have the chain $ z < y < x_1 < \ldots < x_k $ which is impossible.
		
		Thus $ l(X'') < m$. Since $X''$ is infinite, then by the induction hypothesis the lattice $\Con X''$ contains an isomorphic copy of some $\Eq M$ for an infinite set $M$. Since $X''$ is a homomorphic image of the act $X$ then by Lemma~\ref{lemma:01}, $\Con X''$ is a sublattice of lattice $\Con X$. From this, we obtain that the lattice $\Con X$ also embeds $\Eq M$ for an infinite $M$.
	\end{proof}
	
	\section{Acts over completely simple and completely 0-simple semigroups}
	
    A \textit{completely simple} semigroup is a semigroup $S$, which has no non-trivial ideals and has at least one primitive idempotent (i.e. a minimal idempotent with respect to the natural order on the set of idempotents: $e \leqslant f \Leftrightarrow ef=fe=e$). A semigroup $S$ with zero is called \textit{completely 0-simple} if the following conditions are held: 1) $S$ has no ideals other than $\{0\}$ and $S$; 2) $S$ has a 0-minimal (i.e. minimal among non-zero elements) idempotent; 3) $S^2 \neq 0$.
	
	The \textit{Rees matrix semigroup with zero} $\mathcal{M}^0(G,I,\Lambda,P)$ (here $G$ is a group, $I$ and $\Lambda$ are sets, $P$ is an $\Lambda \times I$ matrix with entries $p_{\lambda i} \in G \cup \{ 0\}$ for $\lambda \in \Lambda, i \in I$) is defined as a set consisting of element 0 and elements of the form $(g)_{i\lambda}$, where $g \in G,\ i \in I,\ \lambda \in \Lambda$, with the multiplication
		$$ (g)_{i\lambda} \cdot (h)_{j\mu} =
			\begin{cases}
				(gp_{\lambda j}h)_{i\mu} & \text{if } p_{\lambda j} \neq 0,\\
				0 & \text{if } p_{\lambda j} = 0.
			\end{cases}
		$$
	The \textit{Rees matrix semigroup} $\mathcal{M}(G,I,\Lambda,P)$ where $G,I,\Lambda,P$ is the same as above but $p_{\lambda i} \in G$ for all $i \in I,\lambda \in \Lambda$ is the set of elements of type $(g)_{i\lambda}$ with multiplication $$ (g)_{i\lambda} \cdot (h)_{j\mu} = (gp_{\lambda j}h)_{i\mu}. $$
	
	The well-known Sushkevich -- Rees Theorem states that a semigroup is completely simple iff it is isomorphic to some $\mathcal{M}(G, I,\Lambda,P)$ and a semigroup is completely 0-simple iff it is isomorphic to some $\mathcal{M}^0(G,I,\Lambda,P)$ where the matrix $P$ does not contain zero rows or columns (see~\cite{cliff}, Theorem 3.5 and remarks before Lemma 3.1).
	
	For semigroups $S$ with zero we will consider acts $X$ with zero such that $0 \cdot s = x \cdot 0 = 0$ for all $s\in S,\ x\in X$.
	
	All acts over the semigroup $\mathcal{M}(G,I,\Lambda,P)$ and all acts with zero over the semigroup $\mathcal{M}^0(G,I,\Lambda,P)$ were described in~\cite{avdeev}. Let us give this description, but firstly we will make some preliminary considerations.
	
	Let $S$ be a semigroup with the identity $e$. An act $X$ over $S$ is called \textit{unitary} if $xe=x$ for all $x \in X$. An act $X$ over a semigroup $S$ is called \textit{cyclic} if $X=aS^1$ for some $a \in X$. Every semigroup $S$ is an act over itself, and its congruences are exactly the right congruences of the semigroup. It is easy to verify that a right congruence of a group $G$ is exactly the decomposition into the right cosets of a subgroup of the group $G$. If $H$ is a subgroup of the group $G$, we denote by $G/H$ the set of the right cosets $Hg$ where $g \in G$. Clearly, $G/H$ is an act over $G$ with the operation $Hg \cdot g' = Hgg'$. Let $\{ H_\gamma \mid \gamma \in \Gamma \} $ be a family of subgroups of the group $G$. Put $$ Q = \bigsqcup_{\gamma \in \Gamma} (G/H_\gamma). $$ It is easy to prove that every unitary act over a group G is of the form $ Q = \bigsqcup_{\gamma \in \Gamma} (G/H_\gamma) $ for some family of subgroups $\{ H_{\gamma} \mid \gamma \in \Gamma \}$ of $G$.
	
	The following two propositions describe acts over completely simple and completely 0-simple semigroups.
	
	\begin{proposition}[\cite{avdeev}, Theorem 5] \label{pr01}
	    Let $X$ be a set, $S=\mathcal{M}(G,I,\Lambda,P)$ be a completely simple semigroup, $Q = \bigsqcup_{\gamma \in \Gamma} (G/H_\gamma) $ a unitary act over the group $G$ (where $\{ H_\gamma \mid \gamma \in \Gamma \}$ is a family of subgroups of the group $G$). Define for $ i \in I$ and $\lambda \in \Lambda$  mappings $\pi_i:X \rightarrow Q$, $\varkappa_\lambda: Q \rightarrow X$ such that $q \varkappa_\lambda \pi_i = q \cdot p_{\lambda i}$ for any $q \in Q$, $i \in I$, $\lambda \in \Lambda$. For $x \in X$ and $(g)_{i \lambda} \in S$ put $x \cdot (g)_{i \lambda} = (x \pi_i \cdot g)\varkappa_{\lambda}$. Then $X$ is an act over $S$. Moreover, every act over a completely simple semigroup is isomorphic to an act obtained in this way.
	\end{proposition}
	
	\begin{proposition}[\cite{avdeev}, Theorem 4] \label{pr02}
	    Let $X$ be a set with an element 0, $S=\mathcal{M}^0(G,I,\Lambda,P)$ be a completely 0-simple semigroup, $\{ H_\gamma \mid \gamma \in \Gamma \}$ be a family of subgroups of the group $G$, $ Q = \bigsqcup_{\gamma \in \Gamma} (G/H_\gamma) $ an act over $G$ and $Q^0 = Q \cup \{0\}$. Define for $i \in I$ and $\lambda \in \Lambda$ mappings $\pi_i:X \rightarrow Q^0$, $\varkappa_\lambda: Q^0 \rightarrow X$ such that $0\pi_i = 0$, $0\varkappa_\lambda = 0$, $q \varkappa_\lambda \pi_i = q \cdot p_{\lambda i}$ when $ q \in Q^0$, $i \in I$, $\lambda \in \Lambda$. Put $x \cdot 0 = 0$, $x \cdot (g)_{i \lambda} = (x \pi_i \cdot g)\varkappa_{\lambda}$ when $x \in X$, $(g)_{i \lambda} \in S \setminus \{0\}$. Then $X$ is an act with zero over $S$. Furthermore, any act with zero over a completely 0-simple semigroup is isomorphic to an act obtained in this way.
	\end{proposition}
	
	Let $X$ be an act with zero and let $X_i$ $(i \in I)$ be subacts of $X$. If $X = \bigcup_{i \in I} X_i$ and $X_i \cap X_j = \{0\}$ for $i \neq j$, then we say that $X$ is a 0-coproduct of the acts $X_i$ and write $X = \bigsqcup_{i \in I}^0 X_i$.
	
	Now let $X$ be an act with zero over a completely 0--simple semigroup $S=\mathcal{M}^0(G,I,\Lambda,P)$ obtained by the above described construction, i.e. $Q^0,\varkappa_\lambda,\pi_i$ have the same meaning as in Proposition~\ref{pr02}. Put $Q_\gamma = (G/H_\gamma) \cup \{0\}$ where $\{ H_\gamma \mid \gamma \in \Gamma \}$ is a family of subgroups of the group $G$. For $q \in Q^0$, $\gamma \in \Gamma$ let $X_q = \{q\varkappa_\lambda \mid \lambda \in \Lambda \}$, $X^{(\gamma)} = \bigcup\{X_q \mid q \in Q_{\gamma}\}$. Some properties of these sets are remarked in the next proposition.
	
	\begin{proposition}[\cite{avdeev}, Lemmas 1--4 and Proposition 1] \label{pr03}
		\
		\begin{enumerate}
			\item[(1)] $X^{(\gamma)}$ is a subact of the act $X$;
			\item[(2)] $X^{(\gamma)} \cap X^{(\delta)} = \{0\}$ for $\gamma \neq \delta$;
			\item[(3)] $XS = \bigsqcup_{\gamma \in \Gamma}^0 X^{(\gamma)}$;
			\item[(4)] $xS = X^{(\gamma)}$ for all $x \in X^{(\gamma)} \setminus \{0\}$;
			\item[(5)] $X = (X \setminus XS) \cup \bigsqcup_{\gamma \in \Gamma}^0 z_{\gamma} S $ for some $z_\gamma \neq 0$ and $z_\gamma \in XS$.
		\end{enumerate}
	\end{proposition}
	
	Our next goal is to find for an act $X$ some quotient act of a subact with certain properties in the case when $G$ and $I$ are finite sets, and $\Gamma, \Lambda$ are infinite ones.
	
	Let $S=\mathcal{M}^0(G,I,\Lambda,P)$ be a completely 0-simple semigroup. For $i \in I$ we put $R_i = \{0\} \cup \{(g)_{i\lambda} \mid g \in G, \lambda \in \Lambda\}$. We shall consider $R_i$ as an act over $S$. Obviously $S = \bigsqcup_{i \in I}^0 R_i$.
	
	An act $X$ with zero over a semigroup $S$ with zero is called \textit{0-simple} if $XS \neq \{0\}$ and $X$ has only two subacts: $\{0\}$ and $X$.
	
	\begin{lemma} \label{lemma:A}
	    Let $C$ be a cyclic 0-simple act with zero over a completely 0-simple semigroup $S=\mathcal{M}^0(G,I,\Lambda,P)$ and $CS \neq 0$. Then $C \cong R_i/\rho$ for some $i \in I$ and a congruence $\rho$ of the act $R_i$.
	\end{lemma}
	\begin{proof}
	    Take $c \in C$ such that $cS \neq 0$. Then in view of the 0-simplicity of $C$, $cS=C$. As $S = \bigcup_{i \in I} R_i$ then $cR_i \neq 0$ for some $i \in I$. As $cR_i$ is a non-zero subact then $cR_i = C$. Consider a map $f:R_i \to C$, $r \mapsto cr$. Clearly $f$ is a surjective homomorphism of acts. By the First Isomorphism Theorem $C \cong R_i/\rho$ where $\rho = \text{ker} f$.
	\end{proof}
	
	\begin{remark}
	    It follows from Lemma~\ref{lemma:A} and the formula (5) of Proposition~\ref{pr03} that $XS = \bigsqcup_{\gamma \in \Gamma}^{0} X^{(\gamma)}$ where $X^{(\gamma)}$ = $z_{\gamma}S = z_{\gamma}R_i \cong R_i/\rho$ for some $i \in I$ and $\rho \in \Con R_i$ (here $i$ and $\rho$ depend on $\gamma$).
	\end{remark}
	
	\begin{lemma} \label{lemma:B}
	    Let $S=\mathcal{M}^0(G,I,\Lambda,P)$ be a 0-simple semigroup and $\rho \in \Con R_i$ for some $i \in I$. Then there exists a unique subgroup $H$ of the group $G$ such that
	    \begin{equation} \label{eq1}
	        \forall \lambda \in \Lambda \ \forall g,g' \in G \ \ (g)_{i\lambda} \ \rho \ (g')_{i \lambda} \leftrightarrow g'g^{-1} \in H.
	    \end{equation}
	\end{lemma}
	\begin{proof}
	    Let $(g)_{i \lambda} \ \rho \ (g')_{i \lambda}$. Take $j \in I$ such that $p_{\lambda j} \neq 0$. Then we obtain $(g)_{i \lambda} \cdot (h)_{j \mu} \ \rho \ (g')_{i \lambda} \cdot (h)_{j \mu}$, i.e. $(gp_{\lambda j}h)_{i \mu} \ \rho \ (g' p_{\lambda j} h)_{i \mu}$. This means that $$ (g)_{i \lambda} \ \rho \ (g')_{i \lambda} \leftrightarrow(g)_{i \mu} \ \rho \ (g')_{i \mu} $$ for all $g, g' \in G$, $\lambda, \mu \in \Lambda$. It follows that $$ \rho' = \{ (g,g') \in G \times G \mid (g)_{i \lambda} \ \rho \ (g')_{i \lambda} \ \text{for some (and hence for all)} \ \lambda \in \Lambda \} $$ is a right congruence on the group $G$. Therefore there is a unique subgroup $H \subseteq G$ such that $$ (g,g') \in \rho' \leftrightarrow Hg = Hg'. $$
	\end{proof}
	
	Introduce a relation $\equiv$ on the set $\Lambda$ putting
	\begin{gather}
	    \lambda \equiv \mu \leftrightarrow \forall j\in I \,\,(p_{\lambda j} =0 \leftrightarrow p_{\mu j}=0). \label{eq821}
	\end{gather}
	Let $\rho$ be a congruence on the act $R_i$ such that $\rho \ne \nabla_{R_i}$, and $H$ be a subgroup of $G$ satisfying the condition~\eqref{eq1}. Then we shall call $H$ the subgroup corresponding to $\rho$.

    \begin{lemma} \label{lemma:928}
        Let $H$ be a subgroup of a group $G$ and $S=\mathcal M^0 (G, I, \Lambda, P)$ be a completely 0-simple semigroup, $R_i=\{ (g)_{i\lambda} \mid g\in G, \, \lambda \in \Lambda \} \cup \{ (0) \}$. Put $$ \overline\rho = \{(0,0)\}\cup\{((a)_{i \lambda},(b)_{i\mu}) \mid \lambda \equiv \mu\,\, \& \,\,\forall j\in I \,\, Hap_{\lambda j}= Hbp_{\mu j} \}.$$ Then $\overline \rho$ is a congruence on the act $R_i$ which is the greatest congruence such that $H$ corresponds to it.
    \end{lemma}
    \begin{proof}
        Check that $\overline \rho$ is an equivalence relation. The reflexivity and the symmetry of $\overline \rho$ are obvious. Prove the transitivity. Let $((a)_{i\lambda}, (b)_{i\mu}), ((b)_{i\mu}, (c)_{i\nu}) \in \overline \rho$. As $\lambda \equiv \mu$ and $\mu \equiv \nu$ then $\lambda \equiv \nu$. As $Hap_{\lambda j} = Hbp_{\mu j}$ and $Hbp_{\mu j} = Hcp_{\nu j}$ then $Hap_{\lambda j} = Hcp_{\nu j}$. Thus, $((a)_{i\lambda}, (c)_{i\nu}) \in \overline \rho$.

        Check that $\overline \rho$ is stable under multiplication by elements of $S$. Let $(x, y) =  ((a)_{i\lambda}, (b)_{i\mu}) \in \overline \rho$ and $s= (g)_{k\nu} \in S$. Then $\lambda \equiv \mu$ and $Hap_{\lambda j}= Hbp_{\mu j}$ for all $j\in I$. If $p_{\lambda k}=0$ then $p_{\mu k}=0$, and $xs=ys=0$. Suppose $p_{\lambda k} \ne 0$. Then $xs= (a)_{i\lambda} \cdot (g)_{k\mu} = (ap_{\lambda k}g)_{i\nu}$, $ys= (b)_{i\mu} \cdot (g)_{k\mu} = (bp_{\mu k}g)_{i\nu}$. As $Hap_{\lambda k}gp_{\nu j}= Hbp_{\mu k}gp_{\nu j}$ then $(xs, ys)\in \overline \rho$.

        Prove the maximality of $\overline \rho$. Suppose that $\rho$ is a congruence on $R_i$, $\rho \ne \nabla_{R_i}$ and the subgroup $H$ of $G$ corresponds to $\rho$. As $\rho \ne \nabla_{R_i}$ and $R_i$ is a 0-simple act then 0 is {\it isolated}, i.e. $(0, x)\not\in \rho$ for $x\ne 0$. Let $(x, y)=  ((a)_{i\lambda}, (b)_{i\mu}) \in \rho$. Assume $\lambda \not \equiv \mu$. Then there exists $j\in I$ such that either $p_{\lambda j}=0$, $p_{\mu j}\ne 0$ or $p_{\lambda j} \ne 0$, $p_{\mu j}=0$. We may assume that $p_{\lambda j}=0$, $p_{\mu j}\ne 0$. Put $s= (g)_{j\nu}$, then $xs=0$, $ys \ne 0$. As $(xs,ys) \in \rho$ then $\rho=\nabla_{R_i}$ which is a contradiction. Thus, $\lambda \equiv \mu$.

        Take any $j\in I$. If $p_{\lambda j} =0$ then $p_{\mu j}=0$ and we have $Hap_{\lambda j}= Hbp_{\mu j}=0$. Now suppose that $p_{\lambda j} \ne 0$. We have
        $$ \rho \ni (x, y) \cdot (g)_{j\nu} = ((ap_{\lambda j}g)_{i\nu}, (bp_{\mu j}g)_{i\nu}).  $$
        It follows that $Hap_{\lambda j}g = Hbp_{\mu j}g$, and hence $Hap_{\lambda j} = Hbp_{\mu j}$. Therefore $(x,y) \in \overline{\rho}$ and hence $\rho \subseteq \overline \rho$. Thus, $\overline \rho$ is the greatest congruence for given $H$.
    \end{proof}
	
	\begin{lemma} \label{lemma:456}
	    Let $X$ be a 0-coproduct of infinitely many isomorphic non-zero acts. Then the lattice $\Con X$ does not satisfy a non-trivial lattice identity.
	\end{lemma}
	\begin{proof}
	    Let $X=\coprod_{i\in I}^0X_i$, $I$ be an infinite set. For any $i \in I$ let $\varphi_i: Y\to X_i$ be an isomorphism. For any $\sigma \in \Eq I$ we put $$ \rho(\sigma)= \{ (\varphi_i(y), \varphi_j(y)) \mid (i,j)\in \sigma, y\in Y \}. $$ As $\sigma$ is an equivalence relation and $\varphi_i$ are bijective maps then $\rho(\sigma)$ is a congruence.

        Prove that $\rho(\sigma \cap \tau) = \rho(\sigma) \cap \rho(\tau)$. As $\sigma \cap \tau \subseteq \sigma, \tau$ then $\rho(\sigma \cap \tau) \subseteq \rho(\sigma) \cap \rho(\tau)$. Let $ (x, x') \in \rho(\sigma) \cap \rho(\tau)$. If $x=0$ then $x'=0$ (by definition of $\rho(\sigma)$) and $ (x, x') \in \rho(\sigma \cap \tau)$. Let $x \ne 0$. Then $x' \ne 0$. As $ (x, x') \in \rho(\sigma)$ and $x \ne 0$ then $x=\varphi_i(y)$, $x'=\varphi_j(y)$ for some $y\in Y \setminus \{0\}$, $i,j \in I$ where $(i, j) \in \sigma$. Similarly $(i, j) \in \tau$. Therefore $(i, j)\in \sigma \cap \tau$. It means that $(x, x') \in \rho(\sigma \cap \tau)$.

        Prove that $\rho(\sigma \vee \tau) = \rho(\sigma) \vee \rho(\tau)$.  The inclusion  $\rho(\sigma) \vee \rho(\tau) \subseteq \rho(\sigma \vee \tau)$ is obvious. Let $(x, x') \in \rho(\sigma \vee \tau)$. If $x=0$ then $(x, x') = (0,0) \in \rho(\sigma) \vee \rho(\tau)$. If $x \ne 0$ then $x' \ne 0$ and $(x, x') =(\varphi_i(y), \varphi_j(y))$ for some $y\in Y \setminus \{ 0\}$ and $(i, j) \in \sigma \vee \tau$. Clearly, there exist elements $i_1, i_2, \ldots, i_{2m-1} \in I$ such that $(i, i_1) \in \sigma, \,(i_1, i_2) \in \tau, \, \ldots, \, (i_{2m-2}, i_{2m-1}) \in \sigma, \, (i_{2m-1}, j) \in \tau$. Consider the elements $x_k=\varphi_{i_k}(y)$ $(k=1,2, \ldots, 2m-1)$. Then $(x, x_1) \in \rho(\sigma)$, $(x_1, x_2) \in \rho(\tau)$, $\ldots$,, $(x_{2m-2}, x_{2m-1}) \in \rho(\sigma)$, $(x_{2m-1}, x') \in \rho(\tau)$. We obtain from this that $(x, x') \in \rho(\sigma) \vee \rho(\tau)$. Thus, $\rho(\sigma \vee \tau) = \rho(\sigma) \vee \rho(\tau)$.

        Consider the mapping $\Eq I \to \Con X$, $\sigma \mapsto \rho(\sigma)$. The considerations above show that this mapping is an isomorphic embedding of the lattice $\Eq I$ into the lattice $\Con X$. As $I$ is infinite, then $\Eq I$ does not satisfy a non-trivial lattice identity. Therefore $\Con X$ does not satisfy any too.
	\end{proof}
	
	The next lemma is the last step before the proof of Theorem~\ref{t02}.
	
	\begin{lemma} \label{lemma:C}
	    Let $\rho \neq \nabla_{R_i}$ be a congruence on the act $R_i$ with corresponding subgroup $H \leqslant G$. If $(a)_{i \lambda} \ \rho \ (b)_{i \mu}$ for some $a,b \in G$, $\lambda,\mu \in \Lambda $ then $\lambda \equiv \mu$ (see~\eqref{eq821}) and $\forall j \in I \ p_{\lambda j} \neq 0 \rightarrow p_{\lambda j} p_{\mu j}^{-1} \in a^{-1}Hb$.
	\end{lemma}
	\begin{proof}
	    Let $e$ denote the identity of $G$. Let $(a)_{i \lambda} \ \rho \ (b)_{i \mu}$. If $p_{\lambda j} = 0$ and $p_{\mu j} \neq 0$ then $0 = (a)_{i \lambda} \cdot (e)_{j \lambda} \rho (b)_{i \mu} \cdot (e)_{j \lambda} = (bp_{\mu j})_{i \lambda}$. Thus, $0 \ \rho \ x$ for some $x \neq 0$. As $R_i$ is 0-simple then $\rho = \nabla_{R_i}$, which is impossible. So we proved that $\lambda \equiv \mu$. Let $p_{\lambda j} \neq 0$. Then $(ap_{\lambda j})_{i \lambda} \ \rho \ (bp_{\mu j})_{i \lambda}$. By Lemma~\ref{lemma:B} we obtain $Hap_{\lambda j} = Hbp_{\mu j}$ for some subgroup $H$ i.e. $p_{\lambda j} p_{\mu j}^{-1} \in a^{-1}Hb$.
	\end{proof}
	
	\begin{proof}[Proof of Theorem~\ref{t02}]
		\textit{Sufficiency.} If $X$ is a finite act, then the lattice  $\Con X$ is finite and therefore, satisfies a non-trivial identity.
		
		\textit{Necessity.} Let $X$ be an act with zero over a completely 0-simple semigroup $S={\mathcal M}^0(G,I,\Lambda,P)$, where $|G|,|I| <\infty$ and the lattice $\Con X$ satisfies a non-trivial identity.

        Note that $|X\setminus XS|<\infty$. Indeed, if $X\setminus XS$ is infinite, then the quotient act $X/XS$ is an infinite act consisted only of zeros. Then $\Con (X/XS) = \Eq (X/XS)$. Lemma~\ref{lemma:01} shows that all identities of $\Con X$ are trivial. This contradicts the supposition.

        Thus, $X\setminus XS$ is finite. As $X=(X\setminus XS)\cup \coprod_{\gamma \in \Gamma}^0 z_\gamma S$ (for some $\Gamma$ by Proposition~\ref{pr03}) then it is sufficient to prove the finiteness of $\Gamma$ and every $z_\gamma S$.

        Consider the subact $Y=\coprod_{\gamma \in \Gamma}^0 z_\gamma S$. Seeking a contradiction suppose that $\Gamma$ is infinite. Since $S=\bigcup_{i \in I}R_i$ and $z_\gamma S \ne 0$, there exists $i \in I$ such that $z_\gamma R_i \ne 0$, and therefore, in view of the 0-simplicity of $z_\gamma S$, $z_\gamma S = z_\gamma R_i$. Choose for each $\gamma \in \Gamma$ some $i \in I$ such that $z_\gamma S = z_\gamma R_i \ne 0$. As $I$ is finite and $\Gamma$ is infinite, then there exist an infinite subset $\Gamma_1 \subseteq \Gamma$ and an element $i \in I$ such that $z_\gamma R_i \ne 0$ for all $\gamma \in \Gamma_1$. Consider the subact $Y' =\coprod_{\gamma \in \Gamma_1}^0 z_\gamma S$ of the act $Y$. By Lemma~\ref{lemma:01} $\Con Y'$ is a sublattice of $\Con Y$, therefore $\Con Y'$ satisfies a non-trivial identity.

        We have $z_\gamma S = z_\gamma R_i \ne 0$ for all $\gamma \in \Gamma_1$. As $z_\gamma R_i$ is a cyclic act then $z_\gamma R_i \cong R_i/\rho_\gamma$ for some congruence $\rho_\gamma$ of the act $R_i$. As $z_\gamma R_i \ne 0$ then $\rho_\gamma \ne \nabla_{R_i}$, therefore by Lemma~\ref{lemma:B} there exists a subgroup $H$ of the group $G$ such that $\{ 0 \}$ is a $\rho_\gamma$-class and
        $$ (a)_{i\lambda}\, \rho_\gamma \, (b)_{i\lambda} \leftrightarrow Ha=Hb  $$
        for $a,b\in G$, $\lambda \in \Lambda$. Here $H$ is determined uniquely by $\gamma \in \Gamma_1$. As $G$ is finite then it has only finitely many subgroups, therefore there exist a subgroup $H \subseteq G$ and an infinite subset $\Gamma_2 \subseteq \Gamma_1$ such that~\eqref{eq1} holds for $H$ and for all $\gamma \in \Gamma_2$. By Lemma~\ref{lemma:928} $\rho_\gamma \subseteq \overline \rho_\gamma$ for all $\gamma \in \Gamma$ where $\rho$ is the greatest congruence on $R_i$ for $H$. $\overline \rho_\gamma = \overline\rho_{\gamma'}$ for all $\gamma, \gamma' \in \Gamma_2$. Let $Y''= \coprod_{\gamma \in \Gamma_2}^0 R_i/{\overline \rho_\gamma}$. By Lemma~\ref{lemma:01} the lattice $\Con Y''$ is a sublattice of the lattice $\Con Y'$, therefore the lattice $\Con Y''$ satisfies a non-trivial identity. However, this is impossible because of Lemma~\ref{lemma:456} (as $\Gamma_2$ is infinite and $\overline\rho_\gamma$ coincide with one another). This contradiction shows that $\Gamma$ is finite.

        It remains to prove that each act $z_\gamma S$ is finite. Let suppose that $z_\gamma S$ is infinite. We have $z_\gamma S = z_\gamma R_i$ ($i$ is fixed). The set $z_\gamma R_i$ is infinite. By condition the set $I$ is finite. We may assume that $I=\{ 1,2, \ldots, m\}$. Further, we may connect with any $\lambda \in \Lambda$ a row $r(\lambda) = (p_{\lambda 1}, p_{\lambda 2}, \ldots, p_{\lambda m})$ of elements of a sandwich-matrix $P$. As $G$ is finite then there are only finitely many of different rows. Introduce a relation $\approx$ on the set $\Lambda$ putting

        $$ \lambda \approx \mu \leftrightarrow \forall j\in I \,\, p_{\lambda j} = p_{\mu j} $$
        (i.e. $\lambda \approx \mu \leftrightarrow r(\lambda) = r(\mu)$). Obviously $\approx$ is an equivalence relation and the set $\Lambda$ is divided into finitely many ${\approx}$-classes.

        For any $g \in G$ we put $$ M(g)= \{ z_\gamma \cdot (g)_{i\lambda} \mid \lambda \in \Lambda \}.  $$ Clearly $z_\gamma R_i = \{0\} \cup \bigcup_{g\in G} M(g)$. As $G$ is finite and $z_\gamma R_i$ is infinite then there exist $g\in G$ and infinite subset $\Lambda' \subseteq \Lambda$ such that $z_\gamma \cdot (g)_{ i \lambda } \ne z_\gamma \cdot (g)_{ i \mu}$ for $\lambda, \mu \in \Lambda'$, $\lambda \ne \mu$. Put $u_\lambda=z_\gamma \cdot (g)_{i\lambda}$ for $\lambda \in \Lambda'$. The infinite set $\Lambda'$ is divided by the relation $\approx$ into finitely many classes. Therefore at least one of the classes is infinite, say, $\Lambda''$.

        We have $u_\lambda \ne u_\mu$, $\lambda \approx \mu$ for the elements $\lambda \ne \mu$ of $\Lambda''$. Prove that $u_\lambda s=u_\mu s$ for any $\lambda, \mu \in \Lambda''$ and $s\in S$. Indeed, it is clear for $s=0$. Let $s\ne 0$. Then $s=(h)_{j\nu}$ for some $h\in G$, $j\in I$, $\nu \in \Lambda$. If $p_{\lambda j}=0$, then $p_{\mu j}=0$ (since $\lambda \approx \mu$), therefore $u_\lambda s=u_\mu s=0$. If $p_{\lambda j}\ne 0$, then $p_{\mu j} \ne 0$, moreover $p_{\lambda j}=p_{\mu j}$ (since $\lambda \approx \mu$). Now we have $u_\lambda s=z_\gamma\cdot (g)_{i\lambda} \cdot (h)_{j\nu} = z_\gamma\cdot (gp_{\lambda j}h)_{i\nu} = z_\gamma \cdot (gp_{\mu j}h)_{i\nu} = z_\gamma \cdot (g)_{i\mu} \cdot (h)_{j\nu} =u_\mu s$.

        As $u_\lambda \ne 0$ and $z_\gamma R_i$ is generated by any of its nonzero element, then $u_\lambda R_i = z_\gamma R_i$ for any $\lambda \in \Lambda''$. For any equivalence relation $\sigma \in \Eq \Lambda''$ we construct an equivalence relation $\rho(\sigma)$ on $\Lambda''$ as follows: $$ \rho(\sigma)=\{ (u_\lambda,u_\mu) \mid (\lambda,\mu)\in \sigma \} \cup \Delta_{z_\gamma R_i}.  $$ As $u_\lambda s=u_\mu s$ for all $s\in S$, then $\rho(\sigma)$ is a congruence on $z_\gamma R_i$. It is not difficult to verify that $\rho (\sigma \cap \tau) =\rho(\sigma) \cap \rho(\tau)$ and $\rho (\sigma \vee \tau) =\rho(\sigma) \vee \rho(\tau)$. Moreover, the mapping $\sigma \mapsto \rho(\sigma)$ is injective. Therefore, this mapping is an isomorphic embedding of the lattice $\Eq \Lambda''$ into the lattice $\Con (z_\gamma R_i)$.   Lemma~\ref{lemma:01} shows that the lattice $\Con (z_\gamma R_i)$ does not satisfy a non-trivial identity. But this is impossible. This contradiction completes the proof of the theorem.
	\end{proof}
	
	Recall that an act $U$ over a semigroup $S$ is called \textit{simple} if $uS = U$ for all $ u \in U$, and \textit{congruence-simple} if $|\Con U| \leqslant 2$.
	
	\begin{proof}[Proof of Proposition~\ref{pr01a}]
	    Consider a completely 0-simple semigroup $S = $ \newline $ = \mathcal{M}^0(\{1\},\mathbb{N},\mathbb{N},P)$ where
	    $$ p_{ij} =
			\begin{cases}
				1 \text{ for } i = j,\\
				0 \text{ for } i \neq j.
			\end{cases}
		$$
		Prove that $$ R = \{ (1)_{1i} \mid i \in \mathbb{N} \} \cup \{0\} $$ is an infinite act with zero over $S$, which has a 2-element congruence lattice  $\Con R = \{ \Delta_R, \nabla_R\}$, i.e. $R$ is an congruence-simple act. In this case the lattice satisfies any non-trivial identity.
		
		To prove that $\Con R = \{ \Delta_R, \nabla_R \}$ it is enough to show that any congruence, which is generated by pair $(x,y)$ where $x,y \in R$ and $x \neq y$ coincides with $\nabla_R$.
		
		Let $\rho$ be a congruence generated by the pair $(0,(1)_{1i})$. As $(0,(1)_{1i}) \cdot (1)_{ij} = (0,(1)_{1j})$, then $(0,(1)_{1j}) \in \rho$. Hence $((1)_{1i},(1)_{1j}) \in \rho$. Thus $\rho = \nabla_R$. It is easy to prove that also $\rho = \nabla_R$ if $\rho$ generated by a pair $((1)_{1i},(1)_{1j})$ where $i \neq j$.
	\end{proof}
	
	The following lemmas are preparation for the proof of Proposition~\ref{pr2.1}.
	
	\begin{lemma}   \label{l2.1}
	    Let $U$ be a simple act over a completely simple semigroup $S = \mathcal{M}(G,I,\Lambda,P)$. Then $U = uS = u R_i$ for any $u \in U$, $i \in I$. Also $u \cdot
	    {(p_{\lambda i}^{-1})}_{i \lambda} = u$ for some $\lambda \in \Lambda$.
	\end{lemma}
	\begin{proof}
	    As $uS$ and $uR_i$ are subacts of $U$ and $U$ is simple, then $U = uS = u R_i$. Then $u \in u R_i$, i.e. $u = u \cdot (g)_{i \lambda}$ for some $\lambda \in \Lambda$, $g \in G$. We have $$ u \cdot {(g)}_{i \lambda} = u \cdot (g)_{i \lambda} \cdot {(p_{\lambda i}^{-1})}_{i \lambda} = u \cdot {(p_{\lambda i}^{-1})}_{i \lambda}.$$
	\end{proof}
	
	\begin{lemma} \label{l2.2}
	    Let $U$ be a simple act over a completely simple semigroup $S = \mathcal{M}(G,I,\Lambda,P)$. Then for any $i \in I$ there exists a congruence $\rho$ of an act $R_i$ such that $U \cong {R_i}/{\rho}$.
	\end{lemma}
	\begin{proof}
	    According to the previous Lemma, for some $\lambda \in \Lambda$ we have $u = u e_i$, where $e_i = (p_{\lambda i}^{-1})_{i \lambda}$. Consider the mapping $\varphi: R_i \rightarrow U$, $r \rightarrow ur$, $r \in R_i$. Obviously, $\varphi$ is a homomorphism of acts. Since $U$ is simple, then $R_i \varphi = U$. Now by the Isomorphism Theorem $U \cong {R_i}/{\rho}$ for some $\rho \in \Con R_i$.
	\end{proof}
	
	\par In the article~\cite{oehmke} all right congruences of a completely simple semigroup $S = \mathcal{M}(G,I,\Lambda,P)$ were described. Since $S_S \cong \bigsqcup_{i \in I} R_i$, you can use Theorem 2 from~\cite{oehmke} to find all the congruences of the act $R_i$. However, for further we will not need a full description, but only some properties of congruences on $R_i$.
	
	\begin{lemma} \label{l2.3}
	    Let $\rho$ be a congruence of an act $ R_i$. Consider the act $R_i$ as a disjoint union of subsets: $R_i = \bigcup_{\lambda \in \Lambda}(G)_{i \lambda}$, where $(G)_{i \lambda} = \{ (g)_{i \lambda} \mid g \in G \}$ for $\lambda \in \Lambda$. Then there exists a subgroup $H$ of the group $G$ such that
		\begin{gather}
			((a)_{i \lambda},(b)_{i \lambda}) \in \rho \leftrightarrow Ha = Hb \label{f2.1}
		\end{gather}
		for all $a,b \in G$, $\lambda \in \Lambda$.
	\end{lemma}
	\begin{proof}
	    Fix $\lambda \in \Lambda$ and consider the relation $\rho' = ((G)_{i \lambda} \times (G)_{i \lambda}) \cap \rho$ on the set $(G)_{i \lambda}$. It is easy to see that $(G)_{i \lambda}$ is isomorphic to the group $G$ (the mapping $g \mapsto (g p_{\lambda i}^{-1})_{i \lambda}$ is an isomorphism). Obviously, $\rho'$ is a right congruence on the group $(G)_{i \lambda}$. Let $\sigma$ be a relation on the group $G$ defined by the rule
		\begin{gather}
			(g,g') \in \sigma \leftrightarrow ((g)_{i \lambda},(g')_{i \lambda}) \in \rho. \label{f2.2}
		\end{gather}
		Check that $\sigma$ is a right congruence on the group $G$. Indeed, let $(g,g') \in \sigma$ and $a \in G$. Then it follows from~\eqref{f2.2} that $((g)_{i \lambda},(g')_{i \lambda}) \in \rho$. Multiplying by $(p_{\lambda i}^{-1} a)_{i \lambda}$ we obtain $$ ((g)_{i \lambda} \cdot (p_{\lambda i}^{-1} a)_{i \lambda},(g')_{i \lambda} \cdot (p_{\lambda i}^{-1} a)_{i \lambda}) \in \rho, $$ i.e. $ ((g a)_{i \lambda},(g' a)_{i \lambda}) \in \rho $. Because of~\eqref{f2.2} this means that $(ga,g'a) \in \sigma$. Thus, $\sigma$ is a right congruence on the group $G$. It is well known that any right congruence on the group corresponds to a decomposition of the group $G$ into right cosets by some subgroup $H$. Hence $(g,g') \in \sigma \leftrightarrow Hg = Hg'$.
		
		So, for each $\lambda \in \Lambda$ we have: $$ ((g)_{i \lambda},(g')_{i \lambda}) \in \rho \Leftrightarrow Hg = Hg' $$ for some subgroup $H$ of the group $G$. To prove the statement~\eqref{f2.1} it remains to show that the subgroup $H$ does not depend on $\lambda \in \Lambda$.
		
		\par Let $H,H'$ be subgroups, $\lambda,\mu \in \Lambda$ and
		\begin{gather*}
			((a)_{i \lambda},(b)_{i \lambda}) \in \rho \leftrightarrow Ha = Hb, \\
			((a)_{i \mu},(b)_{i \mu}) \in \rho \leftrightarrow H'a = H'b.
		\end{gather*}
		Let $h \in H$. Then $((e)_{i \lambda},(h)_{i \lambda}) \in \rho$. Multiplying on the right side by $(e)_{i \mu}$, we have $((p_{\lambda i})_{i \mu}, (h p_{\lambda i})_{i \mu}) \in \rho$, which means $H' p_{\lambda i} = H' h p_{\lambda i}$. From here we obtain $h \in H'$. We have proved that $H' \subseteq H$. By symmetry $H \subseteq H'$ too.
	\end{proof}
	
	\begin{lemma} \label{l2.4}
	    Let $S = \mathcal{M}(G,I,\Lambda,P)$ be a completely simple semigroup, $R_i = \{ (g)_{i \lambda} \mid g \in G, \lambda \in \Lambda \} $ be a principal right ideal of the semigroup $S$, considered as a right act over $S$. Let $\rho$ be a congruence on the act $R_i$ and $H$ be a subgroup of the group $G$ defined in Lemma~\ref{l2.3}. If $((a)_{i \lambda},(b)_{i \mu}) \in \rho$, then $p_{\lambda j}p_{\mu j}^{-1} \in a^{-1} H b$ for all $j \in I$.
	\end{lemma}
	\begin{proof}
	    Take any $j \in I$, $\nu \in \Lambda$. Then we have $((a)_{i \lambda} \cdot (e)_{j \nu},(b)_{i \mu} \cdot (e)_{j \nu}) \in \rho$, i.e. $((ap_{\lambda j})_{i \nu},(bp_{\mu j})_{i \nu}) \in \rho$. By Lemma~\ref{l2.3} $H a p_{\lambda j} = H b p_{\mu j}$. Hence, $p_{\lambda j} p_{\mu j}^{-1} \in a^{-1} H b$.
	\end{proof}
	
	\begin{lemma} \label{l2.5}
		Let $S = \mathcal{M}(G,I,\Lambda,P)$ be a completely simple semigroup. Take any $i \in I$ and a subgroup $H$ of the group $G$. For $(a)_{i \lambda},(b)_{i \mu} \in R_i$ we put
		\begin{gather}
			((a)_{i \lambda},(b)_{i \mu}) \in \rho \leftrightarrow \forall j \in I \ p_{\lambda j} p_{\mu j}^{-1} \in a^{-1} H b. \label{f2.3}
		\end{gather}
		Then $\rho$ is a congruence on the act $R_i$. Moreover $\rho$ is the largest congruence on $R_i$, satisfying the condition~\eqref{f2.1}.
	\end{lemma}
	\begin{proof}
		Check that the formula~(\ref{f2.3}) defines a congruence. For $\lambda = \mu$ we have $((a)_{i \lambda},(b)_{i \mu}) \in \rho \leftrightarrow e \in a^{-1} H b$, i.e. $((a)_{i \lambda},(b)_{i \lambda}) \in \rho \leftrightarrow H a = H b$. The reflexivity of the relation $\rho$ is obvious.
		
		Let $((a)_{i \lambda},(b)_{i \mu}) \in \rho$. Then $p_{\lambda j} p_{\mu j}^{-1} \in a^{-1} H b$ for all $j \in I$. Hence, $(p_{\lambda j} p_{\mu j}^{-1})^{-1} \in (a^{-1} H b)^{-1}$, i.e. $p_{\mu j} p_{\lambda j}^{-1} \in b^{-1} H a$ for all $j$. This means that $((b)_{i \mu},(a)_{i \lambda}) \in \rho$. This proves the symmetry of the relation $\rho$. Next, we'll show transitivity.
		
		Let $((a)_{i \lambda},(b)_{i \mu}),((b)_{i \mu},(c)_{i \nu}) \in \rho$. Then $p_{\lambda j} p_{\mu j}^{-1} \in a^{-1} H b$, $p_{\mu j} p_{\nu j}^{-1} \in b^{-1} H c$ for all $j \in I$. Multiplying these relations, we obtain: $p_{\lambda j} p_{\nu j}^{-1} \in a^{-1} H c$. As this is fulfilled for all $j \in I$, then $\rho$ is transitive.
		
		Prove that $\rho$ is stable relative to the multiplication by elements of the semigroup $S$. Let $((a)_{i \lambda},(b)_{i \mu}) \in \rho$. Multiplying it by $(c)_{j \nu}$, we obtain a pair $((ap_{\lambda j}c)_{i \nu},(bp_{\mu j}c)_{i \nu})$. As $p_{\lambda j} p_{\mu j}^{-1} \in a^{-1} H b$, then $H a p_{\lambda j} c = H b p_{\mu j} c$. Hence $ \allowbreak ((a p_{\lambda j} c)_{i \nu},(b p_{\mu j} c)_{i \nu}) \in \rho$.
		
		Prove that the congruence $\rho$ is the largest among the congruences corresponding to the subgroup $H$. Let $\rho' \in \Con R_i$ be such that $((a)_{i \lambda},(b)_{i \lambda}) \in \rho' \leftrightarrow H a = H b$ and let $((a)_{i \lambda},(b)_{i \mu}) \in \rho'$. Then $((a)_{i \lambda} \cdot (e)_{j \nu},(b)_{i \mu} \cdot (e)_{j \nu}) \in \rho'$. That is $((a p_{\lambda j})_{i \nu},(b p_{\mu j})_{i \nu}) \in \rho'$. Therefore $H a p_{\lambda j} = H b p_{\mu j}$, which means $p_{\lambda j} p_{\mu j}^{-1} \in a^{-1}H b$. As this is true for every $j \in I$, then, in view of~(\ref{f2.3}) $((a)_{i \lambda},(b)_{i \mu}) \in \rho$. Thus, $\rho' \subseteq \rho$. This proves the maximality of the congruence $\rho$.
	\end{proof}
	
	\begin{proof}[Proof of Proposition~\ref{pr2.1}]
    	We want to prove that there exists an infinite act $X$ over a semigroup $S = \mathcal{M}(S_3, \mathbb{N}, \mathbb{N}, P)$ such that $\Con X = \{ \Delta, \nabla \}$, by a suitable choice of the sandwich-matrix $P$.
    	
    	Let $G$ be the symmetric group on a 3-element set: $G = S_3 = $ \newline $ =  \{e,(12),(13),(23),(123),(132)\}$. Denote $a = (13),\ b = (23),\ h = (12),\ H = \langle (12) \rangle = \{e,h\}$. We have $I = \Lambda = \mathbb{N}_0 = \{0,1,2,\ldots \}$. The sandwich matrix $P = \Vert p_{\lambda i} \Vert$ will be constructed by the following rule. Consider an $I \times \Lambda$ table $T$
    	\begin{table}[h]
    		\begin{center}
    			\begin{tabular}{cccccccc}
    					&  & & & $\Lambda$ & \\
    					& $ha$ & $b$ & $a$ & $b$ & $a$ & $b$ & $\dots$ \\
    					& $a$ & $hb$ & $a$ & $b$ & $a$ & $b$ & $\dots$ \\
    				$I$	& $a$ & $b$ & $ha$ & $b$ & $a$ & $b$ & $\dots$ \\
    					& $a$ & $b$ & $a$ & $hb$ & $a$ & $b$ & $\dots$ \\
    					& \ldots & \ldots & \ldots & \ldots & \ldots & \ldots & \ldots
    			\end{tabular}
    			\caption{Table $T$} \label{table1}
    		\end{center}
    		that is
    	\end{table}
    	\begin{gather*}
    		t_{j \lambda} =
    		\begin{cases}
    			a&\text{if $\lambda$ is even and } j \neq \lambda,\\
    			ha&\text{if $\lambda$ is even and } j = \lambda,\\
    			b&\text{if $\lambda$ is odd and } j \neq \lambda,\\
    			hb&\text{if $\lambda$ is odd and } j = \lambda.
    		\end{cases}
    	\end{gather*}
    	
    	Define the elements of the matrix $P$ recursively, putting $p_{0j} = e$ for all $j \in \mathbb{N}_0$ and $p_{\lambda + 1,j} = t_{j \lambda}^{-1} p_{\lambda j}$.
    	
    	Let $S = \mathcal{M}(S_3,\mathbb{N}_0,\mathbb{N}_0,P)$, where $P$ is defined above. The right ideal $$ R_0 = \{ (g)_{0 \lambda} \mid g \in G, \lambda \in \Lambda \} $$ of the semigroup $S$ will be considered as an act over $S$. Consider the congruence $\rho$ on $R_0$ defined by the formula~(\ref{f2.3}), which we rewrite in our case as
    	
    	\begin{gather*}
    		((g_1)_{0 \lambda},(g_2)_{0 \mu}) \in \rho \leftrightarrow \forall j \in I \ p_{\lambda j} p_{\mu j}^{-1} \in g_1^{-1} H g_2. \label{f2.4}
    	\end{gather*}
    	Let $X = {R_0}/{\rho}$.
	
	    Firstly prove that $X$ is infinite. It is sufficient to prove that \newline  $((g_1)_{0 \lambda},(g_2)_{0 \mu}) \notin \rho$ for $|\lambda - \mu| \geqslant 2$ and any $g_1,g_2 \in G$. We can assume that $\mu = \lambda + k$, where $k \geqslant 2$. Consider an expression $p_{\lambda j} p_{\mu j}^{-1}$. Transform it to: $p_{\lambda j} p_{\mu j}^{-1} = p_{\lambda j} p_{\lambda + k, j}^{-1} = p_{\lambda j} p_{\lambda + 1, j}^{-1} \cdot p_{\lambda + 1,j} p_{\lambda + 2,j}^{-1} \cdot \ldots \cdot p_{\lambda + k - 1,j} p_{\lambda + k,j}^{-1} = t_{j \lambda}t_{j,\lambda+1} \ldots t_{j, \lambda + k - 1}$. Take the following values of $j$: $j = \lambda, j = \lambda + 1, j = \lambda + 2$. We assume firstly that $\lambda$ is an even number. Consider a fragment of the table $T$:
		\begin{center}
			\begin{tabular}{ccccccc}
					& & $\lambda$ & & & & $\lambda + k $ \\
				$j = \lambda$ & & $ha$ & $b$ & $a$ & $b$ & $\dots$ \\
				$j = \lambda+1$ & & $a$ & $hb$ & $a$ & $b$ & $\dots$ \\
				$j = \lambda+2$ & & $a$ & $b$ & $ha$ & $b$ & $\dots$
			\end{tabular}
		\end{center}
		
		We have: $$p_{\lambda \lambda} p_{\mu \lambda}^{-1} = ha \cdot b \cdot a \cdot w,$$ $$p_{\lambda, \lambda+1} p_{\mu, \lambda+1}^{-1} =a \cdot hb \cdot a \cdot w,$$ $$p_{\lambda, \lambda+2} p_{\mu, \lambda+2}^{-1} =a \cdot b \cdot ha \cdot w,$$ where $w$ is an element of the group $G$. It is easy to check, that elements $hab$, $ahb$ and $abh$ is different. Hence, if $|\lambda - \mu| \geqslant 2$ then the set $\{ p_{\lambda j} p_{\mu j}^{-1} \mid j \in I \}$ contains 3 different elements, therefore $\{ p_{\lambda j} p_{\mu j}^{-1} \mid j \in I \} \not \subseteq g^{-1}_1 H g_2$ (as $|H|=2$). If $\lambda$ is odd, then also $\{ p_{\lambda j} p_{\mu j}^{-1} \mid j \in I \} \not \subseteq g_1^{-1} H g_2$ (as the elements $hba$, $bha$, and $bah$ are different). Thus $X$ is an infinite act.
		
		We only have to prove that $X$ is congruence-simple. By Lemma~\ref{l2.4} $\rho$ is the largest congruence of the act $R_0$ corresponding to the subgroup $H$. Let $\rho' \in \Con R_0$ and $\rho' \supset \rho$. Then $\rho'$ corresponds to some subgroup $H' \supset H$. It is easy to see that $H$ is a maximal proper subgroup of the group $S_3$, therefore $H' = S_3$. Thus,
		\begin{equation}
		    \forall \lambda \in \Lambda \ \forall g_1,g_2 \in G \ ((g_1)_{0 \lambda},(g_2)_{0 \lambda}) \in \rho'. \label{eq43}
		\end{equation}
		
		The table $T$ shows that
		\begin{gather*}
		\{ p_{\lambda j} p_{\lambda+1,j}^{-1} \mid j \in I \} =
			\begin{cases}
				\{a, ha\} & \text{for an even $\lambda$},\\
				\{b, hb\} & \text{for an odd $\lambda$}.
			\end{cases}
		\end{gather*}
		So, for an even $\lambda$ we have: $p_{\lambda j} p_{\lambda+1,j}^{-1} \in Ha$ for all $j \in I$, and for an odd $\lambda$ we have the inclusion $p_{\lambda j} p_{\lambda+1,j}^{-1} \in Hb$ for all $j \in I$. Thus, for any $\lambda \in \Lambda$ there exist $g_1,g_2 \in G$ such that $((g_1)_{0 \lambda},(g_2)_{0,\lambda+1}) \in \rho$.
		Since $\rho' \supset \rho$, then $H' \supset H$. But $H$ is the maximal proper subgroup, hence $H' = G$.
		
		This inclusion together with~\eqref{eq43} yields $\rho' = \nabla_{R_i}$ Returning to the act $X = {R_0}/{\rho}$ we conclude, that the act $X$ has no non-trivial congruences. This means that $\Con X = \{ \Delta_X,\nabla_X \}$.
	\end{proof}
	
	In conclusion, we say a few words about completely simple semigroups $S = \mathcal{M}(G,I,\Lambda,P)$ with an infinite group $G$. Here, infinite acts $X$ with a congruence lattice $\Con X$ satisfying a non-trivial identity exist even in the case when $|I|$, $|\Lambda| < \infty$. Indeed, let $|I| = |\Lambda| = 1$ and $X=G$ be an abelian group. Then the lattice $\Con X$ is modular.
	
	{\bf Acknowledgments}. The authors would like to thank the referee for his valuable comments and suggestions which allowed them to eliminate defects and to correct one proof.
	
	{\bf Data Availability Statement}. No datasets were generated or analysed during the current study.
	
	%\bibliography{bibliography}{}
	%\bibliographystyle{spmpsci}
	
	\begin{thebibliography}{99}

        \bibitem{avdeev} Avdeyev, A.Yu., Kozhuhov, I.B.:
        Acts over completely 0-simple semigroups.
        Acta Cybernetica \textbf{14}(4), 523--531 (2000)

        \bibitem{burris} Burris, S., Sankappanavar, H.P.: A Course in Universal Algebra. Springer, Graduate Texts in Math., 78, New York (1981)

        \bibitem{cliff}
        Clifford, A.H., Preston, G.B.:
        The Algebraic Theory of Semigroups.
        American  Mathematical  Society,  Providence, Rhode  Island (1958)

        \bibitem{kon}
        Cohn, P.M.:
        Universal Algebra.
        Springer Netherlands, Dordrecht (1981)

        \bibitem{free_lattices}
        Freeze R., Je$\tilde{z}$ek J., Nation, J.B.: Free lattices. AMS, Surveys and Monographs, vol. 42 (1995)

        \bibitem{gretz}
        Gr\"atzer, G.:
        Lattice Theory: Foundation.
        Birkh\"auser Verlag, Basel (2011)

        \bibitem{jipsen} Jipsen, P., Rose, H.: Varieties of lattices. Lecture Notes in Mathematics, 1533, Springer-Verlag Berlin, Heidelberg (1992)

        \bibitem{memoirs} Kearnes, K.A., Kiss, E.W.: The shape of congruence lattices. Memoirs of the American Mathematical Society \textbf{222}(1046) (2013)

        \bibitem{hal3}
        Khaliullina, A. R.:
        Modularity conditions of the congruence lattice of acts over right or left zero semigroups.
        Far Eastern Mathematical Journal \textbf{15}(1), 102--120 (2015) (in Russian)

        \bibitem{kilp}
        Kilp, M., Knauer, U., Mikhalev, A.V.:
        Monoids, acts and categories. Walter de Gruyter, Berlin (2000)

       \bibitem{resh}
        Kozhukhov, I.B., Reshetnikov, A.V.:
        Algebras whose equivalence relations are congruences.
        Journal of Mathematical Sciences \textbf{177}(886) (2011)

        \bibitem{oehmke}
        Oehmke, R.H.:
        Right congruences and semisimplicity for Rees matrix semigroups.
        Pacific Journal of Mathematics \textbf{54}(2), 143--164 (1974)

        \bibitem{step}
        Ptahov, D.O., Stepanova, A.A.:
        Congruence lattice of S-acts.
        Far Eastern Mathematical Journal
        \textbf{13}(1), 107--115 (2013) (in Russian)

        \bibitem{sachs}
        Sachs, D.:
        Identities in finite partition lattices.
        Proceedings of the American Mathematical Society
        \textbf{12}, 944--945 (1961)
	
	\end{thebibliography}
	
\end{document}
