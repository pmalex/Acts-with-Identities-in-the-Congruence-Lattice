\documentclass[a4paper]{article}
\usepackage[T2A]{fontenc}
\usepackage[utf8]{inputenc}
\usepackage[english,russian]{babel}
\usepackage{amsmath,amsthm,amssymb}

\bibliographystyle{unsrt}

\title{Полигоны с тождествами в решётке конгруэнций}
\author{И.Б. Кожухов, А.М. Пряничников}
\date{}

\newtheorem{note}{Замечание}
\newtheorem{lemma}{Лемма}
\newtheorem{theorem}{Теорема}
\newtheorem{proposition}{Предложение}
\newtheorem{corollary}{Следствие}

\newcommand{\Con}{\textnormal{Con}\, }
\newcommand{\Eq}{\textnormal{Eq}\, }
\newcommand{\Var}{\textnormal{Var}\, }
\newcommand{\Sub}{\textnormal{Sub}\, }

% \setlist[description]{font=\normalfont}

%%%%%%%%%%%%%%%%%%%%%%%%%%%%%%%%%%%%%%%%%%%%%%%%%%%%%%%

\begin{document}
	
	\maketitle
	
	\section*{Введение}
	
	Решётка конгруэнций $\Con A$ универсальной алгебры $ A $ является важной характеристикой этой алгебры. Наименьшим элементом этой решётки является отношение равенства $ \Delta_A = \{ (a,a) | a \in A \} $, а наибольшим -- универсальное отношение $ \nabla_A = A \times A $. Одним из направлений общей алгебры является изучение универсальных алгебр с теми или иными условиями на конгруэнции. Например, условие тривиальности решётки конгруэнций ($ \Con A = \{ \Delta_A, \nabla_A \} $) определяет простые алгебры (простые группы, кольца, конгруэнц-простые полугруппы и т.д.), условие максимальности или минимальности -- соответственно нётеровы и артиновы алгебры. В работе \cite{resh} исследовался класс алгебр, противоположный классу простых алгебр, а именно, алгебры, у которых всякое отношение эквивалентности является конгруэнцией (т.е. $ \text{Con}A = \text{Eq}A $, где $\Eq A$ -- решётка отношений эквивалентности на множестве $A$). Большое количество работ посвящено подпрямо неразложимым алгебрам, т.е. таким алгебрам $A$, что либо $ |A| = 1 $, либо решётка Con$A$ содержит наименьший отличный от $ \Delta_A $ элемент.
	
	\par Универсальные алгебры, у которых решётка конгруэнций модулярна, или дистрибутивна, или является цепью, тоже привлекали большое внимание специалистов. Можно отметить работы по дистрибутивным и цепным кольцам и модулям, полигонам (над полугруппами) с дистрибутивной или модулярной решёткой конгруэнций \cite{step,hal3}. Интересно отметить, что, хотя полигон над полугруппой -- аналог модуля над кольцом, но решётка конгруэнций модуля (т.е. решётка подмодулей) всегда модулярна, а для решётки конгруэнций полигона модулярность является редким явлением.
	\par Дистрибутивные и модулярные решётки образуют многообразия, задаваемые соответственно тождеством $ (x \vee y) \wedge z = (x \wedge z ) \vee (y \wedge z) $ и тождеством $ (x \vee y) \wedge (x \vee z) = x \vee (z \wedge (x \vee y)) $ (см. \cite[глава 5, теорема 347]{gretz}). Цепи образуют класс решёток, не являющийся многообразием, но замкнутым относительно подрешёток и гомоморфных образов.
	
	\par Пусть $\mathcal{V}$ -- многообразие всех решёток, а $\Var L$ -- многообразие, порождаемое решёткой $L$.
	
	Решёточное тождество называется \textit{нетривиальным}, если оно выполняется не во всех решётках. Обозначим через $FL(n)$ свободную решётку с $n$ свободными образующими. Следующие 3 леммы представляют собой хорошо известные утверждения, их доказательства мы приводим для полноты изложения.
	\begin{lemma} \label{lb0}
	    В конечной решётке выполняется хотя бы одно нетривиальное тождество.
	\end{lemma}
	\begin{proof}
	    Пусть $L$ -- конечная решётка и $\Var L$ -- многообразие, порождённое решёткой $L$. Многообразие, порождённое конечной алгеброй, локально конечно (см. \cite[следствие 3.14]{kon}). Поэтому $\Var L$ состоит из локально конечных алгебр. Вместе с тем, не все решётки локально конечны: результат Ф. Уитмена (см. \cite[\S5, теорема 3]{skorn}) показывает, что для любого $n$ решётка $FL(n)$ изоморфно вкладывается в $FL(3)$, поэтому $FL(n)$ -- бесконечная решётка. Следовательно, $\Var L$ содержит не все решётки, а значит, существует нетривиальное тождество, выполняющееся для всех решёток, содержащихся в $\Var L$, и в частности, для $L$.
	\end{proof}
% 	Известно, что $\mathcal{V}$ порождается конечными решётками (\cite{jipsen}, см. утв. (P1) после теоремы 2.4). Кроме того, по теореме Пудлака -- Тумы, всякая конечная решётка вкладывается в решётку отношений эквивалентности на конечном множестве (см. \cite{Pudlak}). Отсюда мы получаем результат, установленный в 1961 году Д. Саксом (\cite{Sachs}):
	\begin{lemma}[\cite{sachs}] \label{lb1}
		Всякое решёточное тождество, выполняющееся в решётке $\Eq M$ для некоторого бесконечного множества $M$, тривиально.
	\end{lemma}
	
	\begin{lemma} \label{la1}
		Если решётка $L$ содержит в качестве подрешётки $\Eq M$ для какого--либо бесконечного множества $M$, то $\Var L = \mathcal{V}$.
	\end{lemma}
	\begin{proof}
		Если в $L$ выполняется нетривиальное решёточное тождество, то оно должно выполняться и в $\Eq M$, но это не так по лемме \ref{lb1}. Следовательно, в $L$ выполняются только тривиальные решёточные тождества, т.е. $\Var L = \mathcal{V}$.
	\end{proof}
	
	\par В связи с вышесказанным кажется естественным изучение универсальных алгебр, у которых решётка конгруэнций удовлетворяет какому-либо нетривиальному решёточному тождеству. Цель данной работы состоит в доказательстве следующих утверждений.
	
	\begin{theorem} \label{t01}
		Пусть $X$ -- полигон над конечной полугруппой. Тогда решётка конгруэнций $\Con X$ удовлетворяет какому-либо нетривиальному решёточному тождеству в том и только том случае, если $X$ конечен.
	\end{theorem}
	
	\begin{theorem} \label{t02}
		Пусть $S = \mathcal{M}^0(G,I,\Lambda,P)$ -- вполне 0-простая полугруппа и $|G| < \infty,\, |I| < \infty $. Тогда для любого полигона $X$ с нулём над полугруппой $S$ выполняется следующее: решётка конгруэнций $\Con X$ удовлетворяет какому-либо нетривиальному решёточному тождеству в том и только том случае, если $X$ конечен.
	\end{theorem}
	
	Если $I$ -- бесконечное множество, то утверждение теоремы \ref{t02} неверно, как показывает предложение \ref{pr01a}.
	
	\begin{proposition} \label{pr01a}
		Существует вполне 0-простая полугруппа $ S = \mathcal{M}^0(G,I,\Lambda,P) $ и бесконечный полигон $X$ с нулём над $S$ такие, что решётка $\Con X$ двухэлементна (а значит, удовлетворяет нетривиальному решёточному тождеству).
	\end{proposition}
	
	Основные сведения из универсальной алгебры можно найти в \cite{mal,kon}, из теории полугрупп -- в \cite{kliff}, полигонов над полугруппами -- в \cite{klip}, теории решёток -- в \cite{gretz}.
	\par Напомним, что \textit{полигоном над полугруппой} называется множество $X$, на котором действует полугруппа $S$, т.е. определено отображение $ X \times S \rightarrow X,\ (x,s) \mapsto xs $, удовлетворяющее условию $ x(st)=(xs)t $ при всех $x\in X,\ s,t\in S$ (см. \cite{klip}). Полигон можно рассматривать как унарную алгебру, т.е. алгебру, у которой все операции унарны (операциями полигона $X$ над полугруппой $S$ являются умножения на элементы полугруппы, т.е. $ x \mapsto xs \ (s\in S) $).
	\par Если полигон $X$ является объединением своих подполигонов $ X_i\ (i \in I) $ и $ X_i \cap X_j = \emptyset $ при $i \neq j$, то мы называем $X$ \textit{копроизведением} полигонов $X_i$ и пишем $ X = \coprod_{i\in I} X_i $. Для любой полугруппы $S$ мы будем обозначать через $S^1$ наименьшую полугруппу с единицей, содержащую $S$, т.е.
	$$ S^1 = 
		\begin{cases}
			S,\ \text{если $S$ имеет единицу,}\\
			S \cup \{1\},\ \text{если $S$ не имеет единицы.}
		\end{cases}
	$$
	\par Пусть $X$ -- полигон над полугруппой $S$. Для элементов $x,y \in X$ положим $ x \leqslant y \Leftrightarrow y \in xS^1 $. Очевидно, отношение $\leqslant$ является отношением квазипорядка на множестве $X$. Полигон $X$ называется \textit{связным}, если для любых $x,y\in X$ существует последовательность элементов $x_0,x_1,\ldots,x_{2k}\in X$ такая, что $$ x \geqslant x_0 \leqslant x_1 \geqslant x_2 \leqslant \ldots \geqslant x_{2k} \leqslant y. $$ Нетрудно видеть, что связность полигона $X$ над полугруппой $S$ -- это в точности связность графа с множеством вершин $X$ и рёбрами $ (x,xs) $, где $ x \in X,\ s \in S $ и $x \neq xs$. Кроме того, всякий полигон является копроизведением связных подполигонов (компонент связности).
	\par \textit{Нулём} полигона $X$ над полугруппой $S$ назовём такой элемент $z \in X$, что $zs=z$ при всех $s \in S$.
	\par Пусть $X$ -- полигон над полугруппой $S$ и $Y$ -- его подполигон. Конгруэнция $\rho_Y = (Y \times Y) \cup \Delta_X$ наывается \textit{конгруэнцией Риса}. Для фактор-полигона ${X}/{\rho_Y}$ будем использовать также обозначение $X/Y$. Следующее утверждение хорошо известно, доказательство мы приведём лишь для полноты изложения.
	
	\begin{lemma} \label{l01}
		Пусть $X$ -- полигон, $Y$ -- его подполигон. Тогда решётки $\Con Y$ и $\Con {X}/{Y}$ изоморфно вкладываются в решётку $\Con X$.
	\end{lemma}
	\begin{proof}
		Нетрудно проверить, что отображение $\rho \mapsto \rho \cup \Delta_X$ является изоморфным вложением решётки $\Con Y$ в решётку $\Con X$.
	\end{proof}
	Утверждение о том, что $\Con {X}/{Y}$ -- подрешётка решётки $\Con X$, следует из общеалгебраического факта: Если $\rho$ -- конгруэнция алгебры $A$, то существует взаимно однозначное соответствие между конгруэнциями алгебры ${A}/{\rho}$ и теми конгруэнциями на $A$, которые содержат $\rho$; при этом это соответствие сохраняет операции $\vee$ и $\wedge$, поэтому решётка $\Con {A}/{\rho}$ изоморфна отрезку $[\rho, \nabla_A]$ решётки $\Con A$ (см. Биркгоф, гл. VI, теорема 7).
	
	\section*{Полигоны над конечными полугруппами}
	
	Пусть $X$ -- полигон над полугруппой $S$. Ранее был определён квазипорядок на $X$: $ x \leqslant y \Leftrightarrow y \in xS^1 $. По квазипорядку стандартным образом определяются отношения эквивалентности и отношение порядка. А именно, пусть $$ x \sim y \Leftrightarrow x \leqslant y \wedge y \leqslant x. $$ Тогда $\sim$ -- отношение эквивалентности на множестве $X$. На фактор-множестве ${X}/{\sim}$ квазипорядок $\leqslant$ индуцирует порядок, который мы также будем обозначать через $\leqslant$. А именно, пусть $K_1,K_2$ -- два класса эквивалентности отношения $\sim$. $K_1 \leqslant K_2$ означает, что $x \leqslant y$ при каких-либо (а значит, и при всех) $x \in K_1$, $y \in K_2$.
	\par Для $x,y \in X$ полагаем $$ x < y \Leftrightarrow x \leqslant y \wedge x \nsim y. $$
	
	\begin{lemma} \label{l03}
		Отношение $<$ на $X$ транзитивно.
	\end{lemma}
	\begin{proof}
		Пусть $x < y$ и $y < z$. Тогда $x \leqslant y$ и $y \leqslant z$. Ввиду транзитивности отношения $\leqslant$ мы получаем: $x \leqslant z$. Предположим, что $x \sim z$. Тогда  $z \leqslant x$. Отсюда $x \leqslant y \leqslant z \leqslant x$, т.е. $x \sim y$, а это противоречит предположению.
	\end{proof}
	
	\begin{lemma} \label{l04}
		Пусть $X$ -- полигон над конечной полугруппой $S$, причём $|S| = n$. Если $x_k < x_{k-1} < \ldots < x_1 < x_0$ -- последовательность элементов из $X$, то $k \leqslant n$.
	\end{lemma}
	\begin{proof}
		Пусть $k > n$. Мы имеем: $$x_{k-1} = x_k s_{k-1}, x_{k-2} = x_{k-1}s_{k-2}, \ldots , x_1 = x_2 s_1, x_0 = x_1 s_0$$ при некоторых $s_0,\ldots,s_{k-1} \in S^1$. Так как $x_i \neq x_j$ при $i \neq j$, то $s_0,\ldots,s_{k-1} \in S$. Очевидно,
		\begin{gather}
			x_i = x_k s_{k-1} s_{k-2} \ldots s_i \label{lf0}
		\end{gather}
		при всех $i = 0,1,\ldots,k-1$. Элементы $s_{k-1},s_{k-1}s_{k-2},\ldots,s_{k-1}s_{k-2}\ldots s_1s_0$ принадлежат полугруппе $S$, так как $|S| = n$ и $k>n$, то среди этих элементов есть совпадающие, т.е. $s_{k-1}s_{k-2}\ldots s_i = s_{k-1}s_{k-2}\ldots s_j$ при некоторых $i\neq j$. Будем считать, что $i < j$. Из формулы (\ref{lf0}) видно, что в этом случае $x_i = x_j$, однако, это противоречит неравенству $x_j < x_i$.
	\end{proof}
	
	Только что доказанная лемма позволяет ввести понятие длины элемента и длины полигона. Пусть $X$ -- полигон над конечной полугруппой $S$. Положим $$ Z_0 = \{ x \in X \, | \, \forall y \in X \ x \leqslant y \rightarrow x \sim y \}.$$  \textit{Длиной} $l(x)$ \textit{элемента} $x \in X$, назовём наибольшее число $k$ такое, что существует цепочка элементов $$x=x_k < x_{k-1} < \ldots < x_1 < x_0.$$ Так как эта цепочка наибольшей длины, то $x_0 \in Z_0$. По лемме \ref{l04} $k \leqslant n$. \textit{Длиной полигона} $X$ назовём число $l(X) = \max \{ l(x)\, | \, x \in X\}$. Таким образом, $l(x) \leqslant n$ при всех $x \in X$. Очевидно, $l(x) = 0 \Leftrightarrow x \in Z_0$. Ясно, что $l(X) \leqslant n$.
	
	\begin{proof}[Доказательство теоремы \ref{t01}]
		Пусть $X$ -- полигон над конечной полугруппой $S$ и $|S| = n$. Если $X$ -- конечный полигон, то решётка $\Con X$ также конечна, а значит, по лемме \ref{lb0} она содержится в некотором собственном подмногообразии многообразия $\mathcal{V}$.
		\par Осталось доказать, что если $X$ бесконечен, то решётка $\Con X$ не содержится ни в каком собственном подмногообразии многообразия $\mathcal{V}$. Доказательство проведём индукцией по длине $l(X)$ полигона $X$.
		\par \textit{Базис индукции.} Пусть $l(X) = 0$. Тогда $X = Z_0$, и мы имеем: $$ \forall x \in X \ \forall s \in S \ \exists t \in S \ \ xst = x. $$ Очевидно, что в этом случае отношение $\sim$ является конгруэнцией, классами которой являются множества $xS^1\ (x \in X)$. Эти множества конечны, поэтому ${X}/{\sim}$ -- бесконечный полигон, состоящий целиком из нулей. Отсюда следует, что $\Con({X}/{\sim})=\Eq({X}/{\sim})$, и по лемме \ref{lb1} $\Con({X}/{\sim})$ не удовлетворяет никакому нетривиальному решёточному тождеству. Из леммы \ref{l01} следует, что решётка $\Con X$ также не удовлетворяет никакому нетривиальному тождеству.
		\par \textit{Индуктивный переход.} Пусть $l(X) = m > 0$. Если $Z_0$ -- бесконечный подполигон, то по только что доказанному решётка $\Con Z_0$ не содержится ни в каком собственном подмногообразии многообразия $\mathcal{V}$. По лемме \ref{l01} решётка $\Con X$ содержит изоморфную копию решётки $\Con Z_0$, следовательно, решётка $\Con X$ также не содержится ни в каком собственном подмногообразии.
		\par Далее будем считать, что $|Z_0| < \infty$. Положим $X'={X}/{Z_0}$. Тогда $X'$ -- бесконечный полигон.
		\par Пусть $Z_1 = \{x \in X' \ | \ \forall y  \in X' \ x < y \rightarrow y = z_0\}$. Проверим, что $Z_1$ - подполигон. Пусть $x \in Z_1$, $s \in S$. Если $x = z_0$, то $xs = z_0$ (очевидно, $z_0$ -- нуль). Пусть $x \neq z_0$. Имеем: $x \leqslant xs$. Если $xs = z_0$, то $xs \in Z_1$. Если $xs \neq z_0$, то $x \not < xs$. Следовательно, $xs \sim x$. Пусть $xs < y$. Тогда $x \leqslant y$. Если $x < y$, то $y = z_0$ - то, что требуется доказать. Если $x \not < y$, то $x \sim y$, а значит,  $y \sim xs$. Таким образом, $xs \in Z_1$.
		\par Предположим, что $Z_1$ бесконечен. Рассмотрим следующие отношения на полигоне $Z_1$: $$ \alpha = \{ (x,y) \in Z_1 \times Z_1 \ | \ xS^1 = yS^1 \}, $$ $$ \beta = \{ (x,y) \in Z_1 \times Z_1 \ | \ \forall s \in S^1 \ xs = z_0 \leftrightarrow ys = z_0 \}. $$
		Ранее отношение $\alpha$ было обозначено символом $\sim$. Докажем, что отношение $\alpha \cap \beta$ -- отношение эквивалентности. Пусть $s \in S$, $(x,y) \in \alpha \cap \beta$. Если $xs = z_0$, то ввиду включения $(x,y) \in \beta$ мы имеем также $ys = z_0$. Следовательно, $(xs,ys) \in \alpha \cap \beta$. Пусть $xs \neq z_0$. Тогда также $ ys \neq z_0 $. Мы имеем: $x \leqslant xs$. Так как $x \in Z_1$ и $xs \neq z_0$, то $ x \not< xs $. Следовательно, $xs \sim x$. Аналогично $ys \sim y$. Так как $x \sim y$, то $xs \sim ys$, т.е. $(xs,ys) \in \alpha$. Докажем, что $(xs,ys) \in \beta$. Так как $xs \sim ys$, то $xs \cdot 1 = z_0 \leftrightarrow ys \cdot 1 = z_0$. Проверим, что также $xs \cdot t = z_0 \leftrightarrow ys \cdot t = z_0$ при $t \in S$. Пусть $xst = z_0$. Так как $(x,y) \in \beta$, то $yst = z_0$. Теперь ясно, что $(xs,ys) \in \beta$. Таким образом, доказано, что $\alpha \cap \beta$ -- конгруэнция.
		\par Для $s \in S^1$ обозначим через $\rho_S$ отношение эквивалентности, имеющее не более двух классов: $ K_1 = \{ x \in Z_1 \ |\ xs = z_0 \} $, $ K_2 = \{ x \in Z_1 \ |\ xs \neq z_0 \}$ (один из классов может быть пустым). Очевидно, классами отношения эквивалентности $\beta$ являются пересечения классов отношений $\rho_S$. Таким образом, $\beta$ имеет не более, чем $2^{n+1}$ классов. Так как $Z_1$ бесконечно, то существует хотя бы один бесконечный класс, скажем, $K$.
		\par Так как классы отношения $\alpha$ конечны, то то же верно для классов отношения $\alpha \cap \beta$. Следовательно, $Y = {Z_1}/{\alpha \cap \beta}$ -- бесконечный полигон. Для любых $y \in Y$ и $s \in S$ мы имеем: $ys = y$ или $ys = z_0$. Пусть $K'$ -- множество классов отношения $\alpha \cap \beta$, содержащихся в множестве $K$. Очевидно, $K'$ -- бесконечное подмножество полигона $Y$. Возьмём любое $\tau \in \Eq K'$. Положим $\rho(\tau) = \tau \cup \Delta_{Y \setminus K'}$. Проверим, что $\rho(\tau) \in \Con Y$. Действительно, пусть $(y,y') \in \rho(\tau)$ и $s \in S$. Если $(y,y') \notin \tau$, то $y = y'$, а значит, $ys = y's$. Пусть $(y,y') \in \tau$. Тогда $ys \in \{y,z_0\}$, $y' \in \{ y',z_0 \}$, причём $ys = z_0 \leftrightarrow y's = z_0$. Это означает, что либо $(ys.y's) = (y,y')$, либо $(ys,y's) = (z_0,z_0)$. В любом случае $(ys,y's) \in \rho(\tau)$. Нетрудно видеть, что $\{\rho(\tau) \ | \ \tau \in \Eq K' \}$ -- подрешётка решётки $\Con Y$, изоморфная решётке $\Eq K'$. Так как $K'$ бесконечно, то решётка $\Con Y$ не удовлетворяет нетривиальному тождеству.
		\par Чтобы завершить доказательство теоремы, нам осталось рассмотреть случай, когда $Z_1$ -- конечное множество. Положим $X'' = {X'}/{Z_1}$.
		\par Докажем, что $l(X'') < m$. Пусть $$ x_k < x_{k-1} < \ldots < x_1 < x_0 $$ -- цепь наиольшей длины в $X''$. Предположим, что $k \geqslant m$, и приведём это предположение к противоречию. Так как $X'' = (X' \setminus Z_1) \cup \{0\}$, то $x_1 \neq Z_1$. По определению $Z_1$ это означает, что существует такое $y \in X'$, что $x_1 < y$ и $y \neq 0$. Если $y \notin Z_1$, то $y < z$ при некотором $z \neq 0$, и мы можем получить цепочку $$ x_k < x_{k-1} < \ldots < x_1 < y < z $$ элементов из $X$, которая показывает, что $l(X) \geqslant k+1 > m$ -- противоречие с условием. Таким образом, $y \in Z_1$ и $y \neq z_0$ в $X'$, т.е. $y \notin Z_0$ в $X$. Так как $y \notin Z_0$, то $y < z$ при некотором $z \in X$. Мы снова получаем цепочку $$ x_k < x_{k-1} < \ldots < x_1 < y < z, $$ существование которой приводит к противоречию.
		\par Таким образом, $l(X'') < m$. Так как $X''$ бесконечен, то по предположению индукции решётка $\Con X''$ не удовлетворяет нетривиальному решёточному тождеству. Так как $X''$ -- гомоморфный образ полигона $X$, то по лемме \ref{l01} $\Con X''$ -- подрешётка решётки $\Con X$. Отсюда получаем, что решётка $\Con X$ не удовлетворяет нетривиальному тождеству.
	\end{proof}
	
	\section*{Полигоны над вполне простыми и вполне 0-простыми полугруппами}
	
	\textit{Вполне простой} полугруппой называется полугруппа $S$, не имеющая нетривиальных идеалов и имеющая хотя бы один примитивный идемпотент (т.е. идемпотент, минимальный относительно естественного порядка на множестве идемпотентов: $e \leqslant f \Leftrightarrow  ef=fe=e$). Полугруппа $S$ с нулём называется \textit{вполне 0-простой}, если выполнены условия: 1) $S$ не имеет идеалов, отличных от $\{0\}$ и $S$; 2) $S$ имеет 0-минимальный (т.е. минимальный среди ненулевых) идемпотент; 3) $S^2 \neq 0$.
	\par \textit{Рисовская матричная полугруппа} $\mathcal{M}^0(G,I,\Lambda,P)$ (здесь $G$ -- группа, $I$ и $\Lambda$ -- множества, $P=\Vert p_{\lambda i} \Vert$ ($\lambda \in \Lambda, i \in I$) -- матрица с элементами из $G \cup \{0\}$) определяется как множество, состоящее из элемента 0 и элементов вида $(g)_{i\lambda}$, где $g \in G,\ i \in I,\ \lambda \in \Lambda$, с умножением
		$$ (g)_{i\lambda} \cdot (h)_{j\mu} = 
			\begin{cases}
				(gp_{\lambda j}h)_{i\mu},\ \text{если } p_{\lambda j} \neq 0,\\
				0,\ \text{если } p_{\lambda j} = 0.
			\end{cases}
		$$
	\textit{Рисовская матричная полугруппа} $\mathcal{M}(G,I,\Lambda,P)$, где $G,I,\Lambda,P$ такие же, как и выше, но $p_{\lambda i} \in G$ при всех $i \in I,\lambda \in \Lambda$ -- это множество элементов вида $(g)_{i\lambda}$ с умножением $$ (g)_{i\lambda} \cdot (h)_{j\mu} = (gp_{\lambda j}h)_{i\mu}. $$
	
	\par Хорошо известная теорема Сушкевича -- Риса утверждает, что вполне простые полугруппы --- это в точности полугруппы, изморфные полугруппе $\mathcal{M}(G,I,\Lambda,P)$, а вполне 0-простые --- изоморфные рисовской матричной полугруппе $\mathcal{M}^0(G,I,\Lambda,P)$, при условии, что матрица $P$ не содержит нулевых строк или столбцов (см. \cite{kliff}, теорема 3.5 и замечания перед леммой 3.1).
	
	\par Для полугрупп $S$ с нулём мы будем рассматривать полигоны $X$ с нулём и накладывать требование $0 \cdot s = x \cdot 0 = 0$ для любых $s\in S,\ x\in X$.
	
	\par Все полигоны над полугруппой $\mathcal{M}(G,I,\Lambda,P)$ и все полигоны с нулём над полугруппой $\mathcal{M}^0(G,I,\Lambda,P)$ были описаны в работе \cite{avdeev}. Приведём это описание, но сначала надо сделать несколько предварительных рассуждений.
	
	\par Пусть $S$ -- полугруппа с единицей $e$. Полигон $X$ над $S$ называется \textit{унитарным}, если $xe=x$ для всех $x \in X$. Полигон $X$ над полугруппой $S$ (необязательно имеющей единицу) называется \textit{циклическим}, если $X=aS^1$ для некоторого $a \in X$. Всякая полугруппа $S$ является полигоном на собой, при этом конгруэнции этого полигона --- это в точности правые конгруэнции полугруппы. Нетрудно проверить, что правая конгруэнция группы $G$ --- это в точности разложения в правые смежные классы по подгруппе группы $G$. Если $H$ -- подгруппа группы $G$, через $G/H$ мы будем обозначать множество правых смежных классов $Hg$, где $g \in G$. $G/H$ является $G$ -- полигоном относительно операции $Hg \cdot g' = Hgg'$. Пусть $\{ H_\gamma | \gamma \in \Gamma \}$ -- семейство подгрупп группы $G$. Положим $$ Q = \bigsqcup_{\gamma \in \Gamma} (G/H_\gamma). $$ Нетрудно видеть, что написанное выражение -- это общий вид любого унитарного полигона над группой $G$, а $G/H$ -- общий вид унитарного циклического полигона.
	
	\par Следующие два утверждения дают описание полигонов над вполне простыми и вполне 0-простыми полугруппами.
	
	\begin{proposition}[\cite{avdeev}, теорема 5] \label{pr01}
		Пусть $X$ -- множество, $S=\mathcal{M}(G,I,\Lambda,P)$ -- вполне простая полугруппа, $\{ H_\gamma | \gamma \in \Gamma \}$ -- семейство подгрупп группы $G$, $  Q = \bigsqcup_{\gamma \in \Gamma} (G/H_\gamma) $ -- унитарный полигон над $G$. Пусть для $i \in I$ и $\lambda \in \Lambda$ определены отображения $\pi_i:X \rightarrow Q$, $\varkappa_\lambda: Q \rightarrow X$ такие, что $q \varkappa_\lambda \pi_i = q \cdot p_{\lambda i}$ при любых $q \in Q$, $i \in I$, $\lambda \in \Lambda$. Для $x \in X$ и $(g)_{i \lambda} \in S$ положим $x \cdot (g)_{i \lambda} = (x \pi_i \cdot g)\varkappa_{\lambda}$. Тогда $X$ будет являться полигоном над $S$. Кроме того, всякий полигон над вполне простой полугруппой изоморфен полигону, устроенному таким образом.
	\end{proposition}
	
	\begin{proposition}[\cite{avdeev}, теорема 4] \label{pr02}
		Пусть $X$ -- множество с выделенным в нём элементом 0, $S=\mathcal{M}^0(G,I,\Lambda,P)$ -- вполне 0-простая полугруппа, $\{ H_\gamma | \gamma \in \Gamma \}$ -- семейство подгрупп группы $G$, $  Q = \bigsqcup_{\gamma \in \Gamma} (G/H_\gamma) $ -- полигон над $G$. $Q^0 = Q \cup \{0\}$. Пусть для $i \in I$ и $\lambda \in \Lambda$ определены отображения $\pi_i:X \rightarrow Q^0$, $\varkappa_\lambda: Q^0 \rightarrow X$ такие, что $0\pi_i = 0$, $0\varkappa_\lambda = 0$, $q \varkappa_\lambda \pi_i = q \cdot p_{\lambda i}$ при $q \in Q^0$, $i \in I$, $\lambda \in \Lambda$. Положим $x \cdot 0 = 0$, $x \cdot (g)_{i \lambda} = (x \pi_i \cdot g)\varkappa_{\lambda}$ при $x \in X$, $(g)_{i \lambda} \in S \setminus \{0\}$. Тогда $X$ -- полигон с нулём над $S$. Кроме того, любой полигон с нулём над вполне 0-простой полугруппой изоморфен полигону, устроенному таким образом.
	\end{proposition}
	
	\par Пусть $X$ -- полигон с нулём и $X_i$ $(i \in I)$ -- его подполигоны. Если $X = \bigcup_{i \in I} X_i$ и $X_i \cap X_j = \{0\}$ при $i \neq j$, то мы говорим, что $X$ является 0-копроизведением полигонов $X_i$, и пишем $X = \bigsqcup_{i \in I}^0 X_i$.
	
	\par Теперь пусть $X$ -- полигон с нулём над вполне 0-простой полугруппой $S=\mathcal{M}^0(G,I,\Lambda,P)$, полученный вышеописанной конструкцией, т.е. $Q^0,\varkappa_\lambda,\pi_i$ имеют тот же смысл, что и в предложении \ref{pr02}. Положим $Q_\gamma = (G/H_\gamma) \cup \{0\}$. Для $q \in Q^0$, $\gamma \in \Gamma$ положим $X_q = \bigcup\{q\varkappa_\lambda | \lambda \in \Lambda \}$, $X^{(\gamma)} = \bigcup\{X_q | q \in Q\}$. Нам понадобится ряд свойств этих множеств, проверка не составляет труда.
	
	\begin{proposition}[\cite{avdeev}, леммы 1--4 и предл. 1] \label{pr03}
		\
		\begin{enumerate}
			\item[(1)] $X^{(\gamma)}$ -- подполигон полигона $X$;
			\item[(2)] $X^{(\gamma)} \cap X^{(\delta)} = \{0\}$ при $\gamma \neq \delta$;
			\item[(3)] $XS = \bigsqcup_{\gamma \in \Gamma}^0 X^{(\gamma)}$;
			\item[(4)] $xS = X^{(\gamma)}$ для всех $x \in X^{(\gamma)} \setminus \{0\}$;
			\item[(5)] $X = (X \setminus XS) \cup \bigsqcup_{\gamma \in \Gamma}^0 z_{\gamma} S $, причём $z_\gamma \neq 0$ и $z_\gamma \in xS$ при всех $x \in X^{(\gamma)} \setminus \{0\}$.
		\end{enumerate}
	\end{proposition}
	
	\begin{proof}[Доказательство теоремы \ref{t02}]
		\par \textit{Необходимость.} Если $X$ -- конечный полигон, то решётка $\Con X$ конечна, а значит, удовлетворяет нетривиальному решёточному тождеству.
		\par \textit{Достаточность.} Пусть $X$ -- полигон над вполне 0-простой полугруппой $S = \mathcal{M}^0(G,I,\Lambda,P)$, где $|I|,|\Lambda| < \infty$, и решётка $\Con X$ удовлетворяет нетривиальному решёточному тождеству. Заметим вначале, что $|X \setminus XS| < \infty$. Действительно, если $X \setminus XS$ -- бесконечное множество, то ${X}/{XS}$ -- бесконечный полигон с нулевым умножением. Поэтому $\Con ({X}/{XS}) = \Eq ({X}/{XS}).$ Так как ${X}/{XS}$ бесконечно, то по лемме \ref{lb1} $\Eq (X \setminus XS)$ не удовлетворяет нетривиальному тожджеству. По лемме \ref{l01} решётка $\Con X$ содержит подрешётку, изоморфную $\Con ({X}/{XS})$. Следовательно, решётка $\Con X$ не удовлетворяет никакому нетривиальному тождеству, что противоречит предположению.
		\par Итак, $|X \setminus XS| < \infty$. Пользуясь предложением \ref{pr03}, представим $X$ в виде $X = (X \setminus XS) \cup \bigsqcup_{\gamma \in \Gamma}^0 z_{\gamma} S $. Осталось доказать, что $\Gamma$ -- конечное множество и полигоны $z_{\gamma}S$ конечны.
		\par Рассмотрим какое-либо $z_\gamma$. По предожению \ref{pr03} $z_\gamma S$ порождается любым своим ненулевым элементом, в частности, $z_\gamma = z_\gamma s$ при некотором $s \in S$. Так как $z_\gamma \neq 0$, то $s = (g)_{i \lambda}$. Имеем: $z_\gamma = z_\gamma (g)_{i \lambda}$. Найдём такое $j \in I$, что $p_{\lambda j} \neq 0$. Тогда будем иметь: $z_\gamma = z_\gamma \cdot (g)_{i \lambda} = z_\gamma \cdot ((g)_{i \lambda} \cdot (p_{\lambda j}^{-1})_{j \lambda}) = (z_\gamma \cdot (g)_{i \lambda}) \cdot (p_{\lambda j}^{-1})_{j \lambda} = z_\gamma \cdot (p_{\lambda j}^{-1})_{j \lambda}) $. Нетрудно видеть, что $(p_{\lambda j}^{-1})_{j \lambda}$ -- идемпотент. По условию $I$ -- конечное множество. Для каждого $i \in I$ выберем $\lambda \in \Lambda$ так, что $p_{\lambda i} \neq 0$. Положим $e_i = (p_{\lambda i}^{-1})_{i \lambda}$. Таким образом, если $I = \{ i_1,\ldots,i_m \}$, то мы имеем набор идемпотентов $e_1 = (p_{\lambda_1 i_1}^{-1})_{i_1 \lambda_1},\ldots,e_m = (p_{\lambda_m i_m}^{-1})_{i_m \lambda_m}$. Для каждого $\gamma \in \Gamma$ выберем какое-либо одно $i \in I$ такое, что $z_\gamma = z_\gamma e_i$. Соответственно этому $\Gamma$ разобьётся на $m$ подмножеств: $\Gamma_1,\ldots,\Gamma_m$, где $\Gamma_i$ -- множество таких $\gamma$, для которых выбрано $e_i$. Так как $\Gamma$ бесконечно, то хотя бы одно из $\Gamma_i$ также бесконечно. Без ограничения общности можно считать, что $\Gamma_1$ бесконечно. Положим $Y = \bigsqcup_{\gamma \in \Gamma_1	}^0 z_{\gamma} S$. Так как $Y$ -- подполигон полигона $X$, то решётка $\Con Y$ удовлетворяет нетривиальному тождеству.
		\par Проверим, что для $\gamma \in \Gamma_1$ имеет место эквивалентность (при $s \in S$): $$ e_i s = 0 \Leftrightarrow z_\gamma e_is = 0. $$ Действительно, импликация $ \Rightarrow $ очевидна. Докажем обратную импликацию. Пусть $z_\gamma e_is = 0$. Если $e_is \neq 0$, то по свойствам вполне 0-простой полугруппы мы будем иметь $e_i st = e_i$ при некотором $t \in S$. Отсюда получаем: $z_\gamma = z_\gamma e_i = z_\gamma e_ist = z_\gamma e_is \cdot t = 0$, что противоречит выбору элемента $z_\gamma$.
		\par Из только что доказанной эквивалентности слеует ещё одна эквивалентность:
		\begin{gather}
			z_\gamma s = 0 \Leftrightarrow z_\delta s = 0 \label{lf1}
		\end{gather}
		при $\gamma,\delta \in \Gamma_1$ и $s \in S$.
		\par Пусть $\sigma$ -- произвольное отношение эквивалентности на множестве $\Gamma_1$. Обозначим через $\rho(\sigma)$ конгруэнцию полигона $Y$, порождённого парами $(z_\gamma, z_\delta)$, где $(\gamma,\delta) \in \sigma$. Докажем, что отображение $\varphi:\sigma \rightarrow \rho(\sigma)$ является вложением решёток $\varphi:\Eq \sigma \longrightarrow \Con Y$. Тот факт, что $\varphi$ сохраняет решёточные операции $\vee$ и $\wedge$, очевиден. Требуется доказать, что $\varphi$ -- вложение. Для этого покажем, что $(z_\xi,z_\eta) \in \rho(\sigma) \Rightarrow (\xi,\eta) \in \sigma$ при $\xi,\eta \in \Gamma_1$. Так как $(z_\xi,z_\eta) \in \rho(\sigma)$, то мы имеем цепочку равенств
		%\begin{gather*}
		%	z_\xi = u_1s_1, \\ \qquad
		%	v_1s_1 = u_2s_2, \\ \qquad \qquad
		%	v_2s_2 = u_3s_3, \\ \qquad \qquad \qquad
		%	\ldots \\ \qquad \qquad \qquad \qquad \qquad
		%	v_{n-1}s_{n-1} = u_ns_n, \\ \qquad \qquad \qquad \qquad \qquad
		%	v_ns_n = z_\eta,
		%\end{gather*}
		\[ \begin{array}{cccccccccccc}
		    z_\xi & = & u_1s_1, \\
		     & & v_1s_1 & = & u_2s_2, \\
		     & & & & v_2s_2 & = & u_3s_3, \\
		     & & & & & \ldots \\
		     & & & & & & v_{n-1}s_{n-1} & = & u_ns_n, \\
		     & & & & & & & & v_ns_n & = &  z_\eta,
		\end{array} \]
		где $s_1,\ldots,s_n \in S^1$, а $\{ u_i,v_i \} = \{z_{\gamma_i},z_{\delta_i} \}$ при $(\gamma_i,\delta_i) \in \sigma$ $(i = 1,2,\ldots,n)$. Так как $z_\eta \neq 0$, то $u_1s_1 \neq 0$.
		\par Очевидно, $u_1s_1 \neq 0$. Ввиду \ref{lf1} также $v_1s_1 \neq 0$, отсюда $u_2s_2 \neq 0$. И т.д. Получаем, что $u_is_i,v_is_i \neq 0$ при всех $i$. Так как $z_\xi = u_1s_1$ и $u_1 = z_{\xi_1}$, при некотором $\xi_1 \in \Gamma_1$, то $\xi_1 = \xi$. Тогда $v_1 = z_{\eta_1}$ где $(\xi,\eta_1) \in \sigma$. Так как $u_2s_2 \neq 0$ и $u_2 = z_{\xi_2}$, то $\xi_2 = \eta_1$. Продолжим рассуждать аналогичным образом. В результате получим: если $u_i = z_{\xi_i}$, $v_i = z_{\eta_i}$, то $\xi = \xi_1, (\xi_1,\eta_1) \in \sigma$, $\xi_2 = \eta_1$, $(\xi_2,\eta_2) \in \sigma$, $\ldots$ , $\xi_n = \eta_{n-1}$, $(\xi_n,\eta_n) \in \sigma$, $\eta_n = \eta$. Так как $\sigma$ транзитивно, то $(\xi,\eta) \in \sigma$.
		\par Мы доказали, что отображение $\sigma \mapsto \rho(\sigma)$ является вложением решётки $\Eq \Gamma_1$ в решётку $\Con Y$. Наличие такого вложения показывает, что решётка $\Con X$ не удовлетворяет никакому нетривиальному тождеству, а значит, решётка $\Con X$ тоже не удовлетворяет, но это противоречит условию. Полученное противоречие показывает, что $|\Gamma| < \infty$.
		\par Для доказательства теоремы \ref{t02} достаточно показать, что полигоны $z_\gamma S$ конечны. Предположим, что $z_\gamma S$ бесконечен. Так как $z_\gamma \in z_\gamma S$ и $z_\gamma \neq 0$, то $z_\gamma = z_\gamma \cdot (g)_{i \lambda}$ при некоторых $g \in G$, $i \in I$, $\lambda \in \Lambda$. Отсюда получаем: $z_\gamma = z_\gamma \cdot (g)_{i\lambda} \cdot (g)_{i \lambda}$, а значит, $(g)_{i \lambda} \cdot (g)_{i \lambda} \neq 0$, т.е. $p_{\lambda i} \neq 0$. Положим $e_i = (p_{\lambda i}^{-1})_{i \lambda}$. Тогда будем иметь: $z_\gamma = z_\gamma \cdot (g)_{i \lambda} = z_\gamma \cdot ((g)_{i \lambda} \cdot e_i) = (z_\gamma \cdot (g)_{i \lambda})e_i = z_\gamma e_i$. Следовательно, $z_\gamma S = z_\gamma e_i S$.
		\par Очевидно $e_i S = R_i$, где $R_i = \{ (g)_{i \eta} \: | \: g \in G, \eta \in \Lambda \} \cup \{ 0 \}$. Множество $R_i$ -- это правый идеал полугруппы $S$, а значит, $R_i$ -- это правый идеал полугруппы $S$, а значит, $R_i$ -- подполигон полигона $S$, если $S$ рассматривать как полигон над $S$ с естественный действием.
		\par По условию множество $I$ конечно. Будем считать, что $I = \{1,2,\ldots,m\}$. Строка с индексом $\lambda$ сэндвич-матрицы $P$ имеет вид $(p_{\lambda 1},\ldots,p_{\lambda m})$, где $p_{\lambda i} \in G \cup \{0\}$. Так как множество $G$ конечно, то различных строк может быть лишь конечное число. Так как $z_\gamma S = z_\gamma R_i$ -- бесконечное множество, то множество $R_i$ также бесконечно, а значит, $\Lambda$ бесконечно. Разобьём множество $\Lambda$ на классы, относя к одному классу такие $\lambda$ и $\eta$, у которых $(p_{\lambda 1},\ldots,p_{\lambda m}) = (p_{\eta 1},\ldots,p_{\eta m})$. Тогда мы получим разбиение множества $\Lambda$ на конечное число подмножеств: $\Lambda = \Lambda_1 \cup \Lambda_2 \cup \ldots \cup \Lambda_k$. Так как множество $z_\gamma R_i$ бесконечно, то найдётся такое $g \in G$, что множество $\{ z_\gamma (g)_{i \lambda} | \lambda \in \Lambda \}$ бесконечно. Следовательно, можно составить последовательность попарно различных элементов $ z_\gamma (g)_{i \lambda_1}, z_\gamma (g)_{i \lambda_2}, \ldots $.\\
		Пусть $\Lambda' = \{ \lambda_1,\lambda_2, \ldots \}$. Это множество бесконечно. Следовательно, существует такое $t$, что $\Lambda' \cap \Lambda_t$ бесконечно. Множество $\Lambda' \cap \Lambda$ можно рассматривать как подпоследовательность последовательности $\Lambda'$. Пусть $\Lambda' \cap \Lambda_t = \{ \eta_1,\eta_2,\ldots \}$. Тогда мы будем иметь:
		\begin{gather}
			z_\gamma \cdot (g)_{i \lambda_k} \neq z_\gamma \cdot (g)_{i \lambda_l} \text{ при } k \neq l, \label{lf3} \\
			p_{\lambda_k j} = p_{\lambda_l j} \text{ при любых } j \in I \text{ и } k,l \in \mathbb{N}. \label{lf4}
		\end{gather}
		Положим $u_k = z_\gamma \cdot (g)_{i \lambda_k}$ для $k \in \mathbb{N}$. Из \ref{lf3} следует, что $ u_1,u_2,u_3,\ldots $ -- различные элементы полигона $z_\gamma S$. Из \ref{lf4} следует, что $u_k s = u_l s$ при любых $k,l \in \mathbb{N}$ и $s \in S$ Пусть $\sigma$ -- произвольное отношение эквивалентности на множестве $U = \{ u_1,u_2,\ldots \}$. Тогда $\sigma \cup \Delta_{z_\gamma S}$ будет являться конгруэнцией полигона $z_\gamma S$. Таким образом, мы имеем вложение решёток $\Eq U \rightarrow \Con (z_\gamma S)$. Так как множество $U$ бесконечно, то решётка $\Con (z_\gamma S)$ не удовлетворяет нетривиальным решёточным тождествам, а значит, решётка $\Con X$ тоже не удовлетворяет, а это противоречит предположению. Полученное противоречие завершает доказательство теоремы \ref{t02}.
	\end{proof}
	
	\begin{proof}[Доказательство предложения \ref{pr01a}]
		Рассмотрим вполне 0-простую полугруппу $S = \mathcal{M}^0(\{1\},\mathbb{N},\mathbb{N},P)$, где
		$$ p_{ij} =
			\begin{cases}
				1 \text{ при } i = j,\\
				0 \text{ при } i \neq j.
			\end{cases}
		$$
		Докажем, что правый идеал $$ R = \{ (1)_{1i} | i \in \mathbb{N} \} \cup \{0\} $$ является бесконечным полигоном над $S$, у которого решётка конгруэнций двухэлементна, а именно, $\Con R = \{ \Delta_R, \nabla_R\}$, т.е. $R$ -- конгруэнц-простой полигон. В этом случае решётка $\Con R$ конечна, а значит, удовлетворяет нетривиальному тождеству (например, тождеству дистрибутивности).
		\par Для доказательства того, что $\Con R = \{ \Delta_R, \nabla_R \}$, достаточно доказать, что конгруэнция, порождённая парой $(x,y)$, где $x,y \in R$ и $x \neq y$, совпадает с $\nabla_R$.
		\par Пусть $\rho$ -- конгруэнция, порождённая парой $(0,(1)_{1i})$. Так как $(0,(1)_{1i}) \cdot (1)_{ij} = (0,(1)_{1j})$, то $(0,(1)_{1j}) \in \rho$. Аналогично $(0,(1)_{1k}) \in \rho$ при любом $k$. Следовательно, $((1)_{1j},(1)_{1k}) \in \rho$. Таким образом, $\rho = \nabla_R$. Также несложно доказывается, что $\rho = \nabla_R$, если $\rho$ порождается парой $((1)_{1i},(1)_{1j})$, где $i \neq j$.
	\end{proof}
	\newpage
	\section*{Часть 2}
	Полигон $U$ над полугруппой $S$ называется \textit{простым}, если $uS = U$ $\forall u \in U$, и \textit{конгруэнц-простым}, если $|\Con U| \leqslant 2$.
	\begin{lemma}   \label{l2.1}
		Пусть $U$ -- простой полигон над вполне простой полугруппой $S = \mathcal{M}(G,I,\Lambda,P)$. Тогда $U = uS = u R_i$ для любого $u \in U$, $i \in I$. Кроме того, $u \cdot (p_{\lambda i}^{-1})_{i \lambda} = u$ при некотором $\lambda \in \Lambda$.
	\end{lemma}
	\begin{proof}
		Так как $uS$ и $uR_i$ -- подполигоны и $U$ -- простой, то $U = us = u R_i$. Так как $U$ -- простой, то $u \in u R_i$. Следовательно, $u = u \cdot (g)_{i \lambda}$ при некотором $\lambda \in \Lambda$. Имеем: $u \cdot (p_{\lambda i}^{-1})_{i \lambda} = u \cdot (g)_{i \lambda} \cdot  (p_{\lambda i}^{-1})_{i \lambda} = u \cdot (p_{\lambda i}^{-1})_{i \lambda}$.
	\end{proof}
	
	\begin{lemma} \label{l2.2}
		Пусть $U$ -- простой полигон над вполне простой полугруппой $S = \mathcal{M}(G,I,\Lambda,P)$. Тогда для любого $i \in I$ существует такая конгруэнция $\rho$ полигона $R_i$, что $U \cong {R_i}/{\rho}$.
	\end{lemma}
	\begin{proof}
		По предыдущей лемме при некотором $\lambda \in \Lambda$ мы имеем $u = u e_i$, где $e_i = (p_{\lambda i}^{-1})_{i \lambda}$. Рассмотрим отображение $\varphi: R_i \rightarrow U$, $r \rightarrow ur$, $r \in R_i$. Очевидно, $\varphi$ -- гомоморфизм полигонов. Так как $U$ -- простой, то $R_i \varphi = U$. Теперь по теореме об изоморфизме $U \cong {R_i}/{\rho}$ для некоторого $\rho \in \Con R_i$.
	\end{proof}
	
	\par В работе \cite{oehmke} были описаны все правые конгруэнции вполне простой полугруппы $S = \mathcal{M}(G,I,\Lambda,P)$. Так как $S_S \cong \bigsqcup_{i \in I} R_i$, то можно с помощью теоремы 2 из \cite{oehmke} найти все конгруэнции полигона $R_i$. Однако, для дальнейшего нам не нужно будет полное описание, а потребуется лишь установить некоторые свойства конгруэнций на $R_i$. Это мы сделаем независимо от работы \cite{oehmke}.
	
	\begin{lemma} \label{l2.3}
		Пусть $\rho$ -- конгруэнция полигона $R_i$. Полигон $R_i$ рассмотрим как дизъюнктное объединение подмножеств: $R_i = \bigcup_{\lambda \in \Lambda}(G)_{i \lambda}$, где $(G)_{i \lambda} = \{ (g)_{i \lambda} | g \in G \}$ при $\lambda \in \Lambda$. Тогда существует подгруппа $H$ группы $G$ такая, что 
		\begin{gather} 
			((a)_{i \lambda},(b)_{i \lambda}) \in \rho \Leftrightarrow Ha = Hb. \label{f2.1}
		\end{gather}
	\end{lemma}
	\begin{proof}
		Зафиксируем $\lambda \in \Lambda$ и рассмотрим отношение $\rho' = ((G)_{i \lambda} \times (G)_{i \lambda}) \cap \rho$ на множестве $(G)_{i \lambda}$. Нетрудно видеть, что $(G)_{i \lambda}$ -- группа, изоморфная группе $G$ (изоморфизмом является отображение $g \mapsto (g p_{\lambda i}^{-1})_{i \lambda}$). Очевидно, $\rho'$ является правой конгруэнцией группы $(G)_{i \lambda}$. Пусть $\sigma$ -- отношение на группе $G$, определённое правилом 
		\begin{gather}
			(g,g') \in \sigma \Leftrightarrow ((g)_{i \lambda},(g')_{i \lambda}) \in \rho. \label{f2.2}
		\end{gather}
		Проверим, что $\sigma$ -- правая конгруэнция группы $G$. Действительно, пусть $(g,g') \in \sigma$ и $a \in G$. Тогда из \ref{f2.2} следует, что $((g)_{i \lambda},(g')_{i \lambda}) \in \rho$. Умножив на $(p_{\lambda i}^{-1} a)_{i \lambda}$ получим: $$ ((g)_{i \lambda} \cdot (p_{\lambda i}^{-1} a)_{i \lambda},(g')_{i \lambda} \cdot (p_{\lambda i}^{-1} a)_{i \lambda}) \in \rho, $$ т.е. $ ((g a)_{i \lambda},(g' a)_{i \lambda}) \in \rho $. Ввиду \ref{f2.2} это означает, что $(ga,g'a) \in \sigma$. Таким образом, $\sigma$ -- правая конгруэнция группы $G$. Хорошо известно, что правая конгруэнция на группе соответствует разложению группы $G$ на правые смежные классы по некоторой подгруппе $H$. Следовательно, $(g,g') \in \sigma \Leftrightarrow Hg = Hg'$.
		\par Итак, для каждого $\lambda \in \Lambda$ мы имеем: $$ ((g)_{i \lambda},(g')_{i \lambda}) \in \rho \Leftrightarrow Hg = Hg' $$ для некоторой подгруппы $H$ группы $G$. Для доказательства утверждения \ref{f2.1} осталось показать, что подгруппа $H$ не зависит от $\lambda \in \Lambda$.
		\par Пусть $H,H'$ -- подгруппы, $\lambda,\mu \in \Lambda$ и 
		\begin{gather*}
			((a)_{i \lambda},(b)_{i \lambda}) \in \rho \Leftrightarrow Ha = Hb, \\
			((a)_{i \mu},(b)_{i \mu}) \in \rho \Leftrightarrow Ha' = Hb'.
		\end{gather*}
		Пусть $h \in H$. Тогда $((e)_{i \lambda},(h)_{i \lambda}) \in \rho$. Умножив справа на $(e)_{i \mu}$, получим: $((p_{\lambda i})_{i \mu}, (h p_{\lambda i})_{i \mu}) \in \rho$, а значит, $H' p_{\lambda i} = H' h p_{\lambda i}$. Отсюда получаем: $h \in H'$. Нами доказано, что $H' \subseteq H$.
	\end{proof}
	
	\begin{lemma} \label{l2.4}
		Пусть $S = \mathcal{M}(G,I,\Lambda,P)$ -- вполне простая полугруппа, $R_i = \{ (g)_{i \lambda} | g \in G, \lambda \in \Lambda \}$ -- главный правый идеал полугруппы $S$, рассматриваемый как правый полигон над $S$. Пусть $\rho$ -- конгруэнция полигона $R_i$ и $H$ -- подгруппа группы $G$, определённая в лемме \ref{l2.3}. Если $((a)_{i \lambda},(b)_{i \mu}) \in \rho$, то $p_{\lambda j}p_{\mu j}^{-1} \in a^{-1} H b$ при всех $j \in I$.
	\end{lemma}
	\begin{proof}
		Возьмём любые $j \in I$, $\nu \in \Lambda$. тогда получим: $((a)_{i \lambda} \cdot (e)_{j \nu},(b)_{i \mu} \cdot (e)_{j \nu}) \in \rho$, т.е. $((ap_{\lambda j})_{i \nu},(bp_{\mu j})_{i \nu}) \in \rho$. По лемме \ref{l2.3} $H a p_{\lambda i} = H b p_{\mu j}$. Следовательно, $p_{\lambda j} p_{\mu j}^{-1} \in a^{-1} H b$.
	\end{proof}
	
	\begin{lemma} \label{l2.5}
		Пусть $S = \mathcal{M}(G,I,\Lambda,P)$ -- вполне простая полугруппа. Возьмём любое $i \in I$ и подгруппу $H$ группы $G$. Для $(a)_{i \lambda},(b)_{i \mu} \in R_i$ положим
		\begin{gather}
			((a)_{i \lambda},(b)_{j \mu}) \in \rho \Leftrightarrow \forall j \in I \ p_{\lambda j} p_{\mu j}^{-1} \in a^{-1} H b. \label{f2.3}
		\end{gather}
		\par Тогда $\rho$ является конгруэнцией полигона $R_i$. Кроме того, $\rho$ -- наибольшая конгруэнция на $R_i$, удовлетворяющая условию \ref{f2.1}.
	\end{lemma}
	\begin{proof}
		Проверим, что формула \ref{f2.3} определеяет конгруэнцию. При $\lambda = \mu$ мы получим $((a)_{i \lambda},(b)_{i \mu}) \in \rho \Leftrightarrow e \in a^{-1} H b$, то есть $((a)_{i \lambda},(b)_{i \lambda}) \in \rho \Leftrightarrow H a = H b$. Поэтому на каждом множестве $(G)_{i \lambda}$ полигона $R_i$ мы получаем разбиение группы $G$ на правые смежные классы по $H$. Отсюда следует рефлексивность отношения $\rho$.
		\par Пусть $((a)_{i \lambda},(b)_{i \mu}) \in \rho$. Тогда $p_{\lambda j} p_{\mu j}^{-1} \in a^{-1} H b$ при всех $j \in I$. Следовательно, $(p_{\lambda j} p_{\mu j}^{-1})^{-1} \in a^{-1} H b$, т.е. $p_{\mu j} p_{\lambda j}^{-1} \in b^{-1} H a$ при всех $j$. Это означает, что $((b)_{i \mu},(a)_{i \lambda}) \in \rho$. Этим доказана симметричность отношения $\rho$. Покажем транзитивность.\\
		Пусть $((a)_{i \lambda},(b)_{i \mu}),((b)_{i \mu},(c)_{i \nu}) \in \rho$. Тогда $p_{\lambda j} p_{\mu j}^{-1} \in a^{-1} H b$, $p_{\mu j} p_{\nu j}^{-1} \in b^{-1} H c$ при всех $j \in I$. Перемножив эти соотношения, получим: $p_{\lambda j} p_{\nu j}^{-1} \in a^{-1} H c$. Так как это выполняется при всех $j \in I$, то $\rho$ транзитивно.
		\par Докажем, что $\rho$ выдерживает умножение на элементы полугруппы $S$. Пусть $((a)_{i \lambda},(b)_{i \mu}) \in \rho$. Умножив на $(c)_{j \nu}$, получим пару $((ap_{\lambda j})_{i \nu},(bp_{\mu j})_{i \nu})$. Так как $p_{\lambda j} p_{\mu j}^{-1} \in a^{-1} H b$, то $H a p_{\lambda j} = H b p_{\mu j}$. Следовательно, $((a p_{\lambda j})_{i \nu},(b p_{\mu j})_{i \nu}) \in \rho$.
		\par Докажем, что конгруэнция $\rho$ наибольшая среди тех, которые соответствуют подгруппе $H$. Пусть $\rho' \in \Con R_i$ такова, что $((a)_{i \lambda},(b)_{i \lambda}) \in \rho \Leftrightarrow H a = H b$ и пусть $((a)_{i \lambda},(b)_{i \mu}) \in \rho'$. Тогда $((a)_{i \lambda} \cdot (e)_{j \nu},(b)_{i \mu} \cdot (e)_{j \nu}) \in \rho'$. То есть $((a p_{\lambda j})_{i \nu},(b p_{\mu j})_{i \nu}) \in \rho'$. Поэтому $H a p_{\lambda j} = H b p_{\mu j}$, а значит, $p_{\lambda j} p_{\mu j}^{-1} \in H$, так как это выполнено для всех $j \in I$, то ввиду (\ref{f2.3}) $((a)_{i \lambda},(b)_{i \mu}) \in \rho$. Таким образом, $\rho' \subseteq \rho$. Этим доказана максимальность конгруэнции $\rho$.
	\end{proof}
	
	\par Покажем теперь, что существует конечная группа $G$ и бесконечные множества $I$ и $\Lambda$ такие, что над полугруппой $S = \mathcal{M}(G,I,\Lambda,P)$ при подходящем выборе сэндвич-матрицы $P$ можно построить бесконечный полигон $X$ над $S$, у которого решётка конгруэнций тривиальна, т.е. $\Con X = \{\Delta_X, \nabla_X\}$, а значит, решётка $\Con X$ удовлетворяет нетривиальному тождеству.
	\par Возьмём в качестве группы $G$ симметрическую группу на 3-элементном множестве: $G = S_3 = \{e,(12),(13),(23),(123),(132)\}$. Положим $a = (13),\ b = (23),\ h = (12),\ H = \langle (12) \rangle = \{e,h\}$. Далее, пусть $I = \Lambda = \mathbb{N}_0 = \{0,1,2,\ldots \}$. Сэндвич-матрицу $P = \Vert p_{\lambda i} \Vert$ будем строить из следующих соображений. Рассмотрим $I \times \Lambda$ таблицу $T$
	\begin{table}[h]
		\begin{center}
			\begin{tabular}{cccccccc}
					&  & & & $\Lambda$ & \\
					& $ha$ & $b$ & $a$ & $b$ & $a$ & $b$ & $\dots$ \\
					& $a$ & $hb$ & $a$ & $b$ & $a$ & $b$ & $\dots$ \\
				$I$	& $a$ & $b$ & $ha$ & $b$ & $a$ & $b$ & $\dots$ \\
					& $a$ & $b$ & $a$ & $hb$ & $a$ & $b$ & $\dots$ \\
					& \ldots & \ldots & \ldots & \ldots & \ldots & \ldots & \ldots
			\end{tabular}
			\caption{Таблица $T$} \label{table1} 
		\end{center}
		то есть
	\end{table}
	\begin{gather*}
		t_{j \lambda} = 
		\begin{cases}
			a,\ \text{если $\lambda$ чётно и } j \neq \lambda,\\
			ha,\ \text{если $\lambda$ чётно и } j = \lambda,\\
			b,\ \text{если $\lambda$ нечётно и } j \neq \lambda,\\
			hb,\ \text{если $\lambda$ нечётно и } j = \lambda.
		\end{cases}
	\end{gather*}
	\par Элементы матрицы $P$ определим рекуррентно, полагая $p_{0j} = e$ для всех $j \in \mathbb{N}_0$ и $p_{\lambda + 1,j} = t_{j \lambda}^{-1} p_{\lambda j}$ $(\lambda,j \in \mathbb{N}_0)$.
	\par Пусть $S = \mathcal{M}(S_3,\mathbb{N}_0,\mathbb{N}_0,P)$, где $P$ определена выше. Правый идеал $$ R_0 = \{ (g)_{0 \lambda} | g \in G, \lambda \in \Lambda \} $$ полугруппы $S$ будем рассматривать как полигон над $S$. Рассмотрим конгруэнцию $\rho$ над $R_0$, определённую по формуле \ref{f2.3}, которую мы в нашем случае перепишем в виде
	\begin{gather}
		((g_1)_{0 \lambda},(g_2)_{0 \mu}) \in \rho \Leftrightarrow \forall j \in I \ p_{\lambda j} p_{\mu j}^{-1} \in g_1^{-1} H g_2. \label{f2.4}
	\end{gather}
	Положим $X = {R_0}/{\rho}$.
	\begin{proposition} \label{pr2.1}
		$X$ -- бесконечный полигон над полугруппой $S = \mathcal{M}(S_3,\mathbb{N}_0,\mathbb{N}_0,P)$, причём $\Con X = \{ \Delta_X, \nabla_X \}$.
	\end{proposition}
	\begin{proof}
		Докажем вначале, что $X$ бесконечен. Для этого достаточно доказать, что $((g_1)_{0 \lambda},(g_2)_{0 \mu}) \notin \rho$ при $|\lambda - \mu| \geqslant 2$ и любых $g_1,g_2 \in G$. Можно считать, что $\mu = \lambda + k$, где $k \geqslant 2$. Рассмотрим выражение $p_{\lambda j} p_{\mu j}^{-1}$. Преобразуем его: $p_{\lambda j} p_{\mu j}^{-1} = p_{\lambda j} p_{\lambda + k, j}^{-1} = p_{\lambda j} p_{\lambda + 1, j}^{-1} \cdot p_{\lambda + 1,j} p_{\lambda + 2,j}^{-1} \cdot \ldots \cdot p_{\lambda + k - 1,j} p_{\lambda + k,j}^{-1} = t_{j \lambda}t_{j,\lambda+1} \ldots t_{j, \lambda + k - 1}$. Возьмём следующие значения индекса $j$: $j = \lambda, j = \lambda + 1, j = \lambda + 2$. Будем считать, что $\lambda$ -- чётное число (случай нечётного $\lambda$ рассматривается аналогично). Рассмотрим фрагмент таблицы $T$:
		\begin{center}
			\begin{tabular}{ccccccc}
					& & $\lambda$ & & & & $\lambda + k $ \\
				$j = \lambda$ & & $ha$ & $b$ & $a$ & $b$ & $\dots$ \\
				$j = \lambda+1$ & & $a$ & $hb$ & $a$ & $b$ & $\dots$ \\
				$j = \lambda+2$ & & $a$ & $b$ & $ha$ & $b$ & $\dots$
			\end{tabular}
		\end{center}
		Имеем: $$p_{\lambda \lambda} p_{\mu \lambda}^{-1} = ha \cdot b \cdot a \cdot w,$$ $$p_{\lambda, \lambda+1} p_{\mu, \lambda+1}^{-1} =a \cdot hb \cdot a \cdot w,$$ $$p_{\lambda, \lambda+2} p_{\mu, \lambda+2}^{-1} =a \cdot b \cdot ha \cdot w,$$ где $w$ -- некоторый элемент группы $G$. Нетрудно проверить, что элементы $haba$, $ahba$ и $abha$ различны. Следовательно, при $|\lambda - \mu| \geqslant 2$ множество $\{ p_{\lambda j} p_{\mu j}^{-1} | i \in I \}$ содержит 3 различных элемента, поэтому $\{ p_{\lambda j} p_{\mu j}^{-1} | i \in I \} \not \subseteq \rho$. таким образом, $X$ -- бесконечный полигон.
		\par Докажем теперь, что $X$ конгруэнц-простой. По лемме \ref{l2.4} $\rho$ -- наибольшая конгруэнция полигона $R_0$, соответствующая подгруппе $H$. Пусть $\rho' \in \Con R_0$ и $\rho' \supset \rho$. Тогда $\rho'$ соответствует некоторой подгруппе $H' \supset H$. Нетрудно видеть, что $H$ -- максимальная собственная  подгруппа группы $S_3$, поэтому $H' = S_3$.
		\par Из таблицы $T$ видно, что
		\begin{gather*}
		\{ p_{\lambda j} p_{\lambda+1,j}^{-1} | j \in I \} = 
			\begin{cases}
				\{a, ha\} \ \text{при чётном $\lambda$},\\
				\{b, hb\}\ \text{при нечётном $\lambda$}.
			\end{cases}
		\end{gather*}
		Значит, при чётном $\lambda$ мы имеем: $p_{\lambda j} p_{\lambda+1,j}^{-1} \in Ha$ при всех $j \in I$, а при нечётном $\lambda$ имеем включение $p_{\lambda j} p_{\lambda+1,j}^{-1} \in Hb$ при всех $j \in I$. Таким образом, для любого $\lambda \in \Lambda$ существуют $g_1,g_2 \in G$ такие, что $((g_1)_{0 \lambda},(g_2)_{0 \lambda+1}) \in \rho$. Так как $\rho' \supset \rho$ и $\rho$ -- наибольшая конгруэнция, соответствующая подгруппе $H$, то по лемме \ref{l2.4} существуют элементы $\lambda \in \Lambda$ и $g_1,g_2 \in G$ такие, что $((g_1)_{0 \lambda},(g_2)_{0 \lambda}) \in \rho'$ и $Hg_1 \neq Hg_2$. Лемма \ref{l2.3} показывает теперь, что существует подгруппа $H'$ группы $G$ такая, что $$ ((g_1)_{0 \lambda},(g_2)_{0 \lambda}) \in \rho' \Leftrightarrow H'a = H'b. $$ Так как $\rho' \supset \rho$, то $H' \supset H$. Но $H$ -- максимальная собственная подгруппа, следовательно, $H' = G$. Таким образом, $((g_1)_{0 \lambda},(g_2)_{0 \lambda}) \in \rho'$ при любых $\lambda \in \Lambda$ и $g_1,g_2 \in G$. Ранее мы доказали, что $((g_1)_{0 \lambda},(g_2)_{0, \lambda+1}) \in \rho$ при любом $\lambda$ и подходящем выборе элементов $g_1,g_2 \in G$. Подводя итог этим рассуждениям, мы получаем, что $((g_1)_{0 \lambda},(g_2)_{0 \mu}) \in \rho'$ при любых $\lambda,\mu \in \Lambda$ и $g_1,g_2 \in G$. Таким образом, $\rho' = \nabla_{R_0}$. Переходя к полигону $X = {R_0}/{\rho}$, мы заключаем, что полигон $X$ не имеет нетривиальных конгруэнций. То есть $\Con X = \{ \Delta_x,\nabla_x \}$.
		
		\begin{corollary}
			Существует вполне простая полугруппа $S = \mathcal{M}(G,I,\Lambda,P)$ с конечной группой $G$ и бесконечными множествами $I$ и $\Lambda$ и бесконечный полигон $X$ над $S$ такой, что решётка $\Con X$ не удовлетворяет никакому нетривиальному тождеству. 
		\end{corollary}
		
		\par В заключение скажем несколько слов о вполне простых полугруппах $S = \mathcal{M}(G,I,\Lambda,P)$, у которых группа $G$ бесконечна. Здесь бесконечные полигоны $X$ с решёткой конгруэнций $\Con X$, удовлетворяющей нетривиальному тождеству, существуют даже в случае когда $|I|$, $|\Lambda| < \infty$. Действительно, пусть $|I| = |\Lambda| = 1$ и $G = \mathbb{Z}_{p^\infty}$ -- квазициклическая группа. Тогда $S \cong G$, и решётка конгруэнций полигона $S$ (над $S$) изоморфна решётке  подгрупп группы $G$. Хорошо известно, что решётка $\Sub \mathbb{Z}_{p^\infty}$ является бесконечной цепью, а значит, удовлетворяет нетривиальному тождеству (скажем, тождеству дистрибутивности). Впрочем, любая абелева группа имеет модулярную решётку конгруэнций, а значит, в ней выполняется нетривиальное тодлество.
	\end{proof}
	\bibliography{bibliography}
	
\end{document}
